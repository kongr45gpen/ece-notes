\documentclass[11pt,a4paper,titlepage,fleqn]{article}

\usepackage{amsmath}
\usepackage{amsfonts}
\usepackage{amssymb}
\usepackage{libs/commath2}
\usepackage[table]{xcolor}
\usepackage[hidelinks,draft=false]{hyperref}
\usepackage[skins,theorems]{tcolorbox}
\usepackage{titlesec}
\usepackage{tikz}
\usepackage{libs/circuitikz} % use our own recent version to make sure some bugs are fixed
\usepackage{pgfplots}
\usepackage{mathtools}
\usepackage[makeroom]{cancel}
\usepackage{mathrsfs}
\usepackage{wrapfig}
%\usepackage{subcaption}
%\usepackage{floatrow}
\usepackage{esint}
\usepackage{enumitem}
%\usepackage{bm}
\usepackage{relsize}
\usepackage{xfrac}
\usepackage{comment}
\usepackage{siunitx}
%\usepackage{MnSymbol}
\usepackage[obeyDraft,disable]{todonotes}
%\usepackage[linesnumbered,lined]{algorithm2e}


\pgfplotsset{compat=1.13}
\usetikzlibrary{arrows.meta}
\usetikzlibrary{patterns}
\usetikzlibrary{decorations.pathmorphing,patterns}
\usetikzlibrary{decorations.markings}
\usetikzlibrary{backgrounds}
\usetikzlibrary{shapes.misc}
\usetikzlibrary{shapes.multipart}
\usetikzlibrary{shadows.blur}
\usetikzlibrary{fadings}
\usetikzlibrary{intersections}
\usetikzlibrary{arrows.meta}
\usetikzlibrary{calc}
\usetikzlibrary{matrix}
\usetikzlibrary{positioning}
\usetikzlibrary{shapes}
\usetikzlibrary{shadings}

\tcbuselibrary{breakable}

\tikzset{cross/.style={cross out, draw,
        minimum size=2*(#1-\pgflinewidth),
        inner sep=0pt, outer sep=0pt}}
\tikzset{
    mark position/.style args={#1(#2)}{
        postaction={
            decorate,
            decoration={
            	post length=1mm, % ??? Magic to fix "Dimension
            	pre length=1mm, % ???  too large" errors.
                markings,
                mark=at position #1 with \coordinate (#2);
            }
        }
    }
}
\makeatletter
\tikzset{
  use path for main/.code={%
    \tikz@addmode{%
      \expandafter\pgfsyssoftpath@setcurrentpath\csname tikz@intersect@path@name@#1\endcsname
    }%
  },
  use path for actions/.code={%
    \expandafter\def\expandafter\tikz@preactions\expandafter{\tikz@preactions\expandafter\let\expandafter\tikz@actions@path\csname tikz@intersect@path@name@#1\endcsname}%
  },
  use path/.style={%
    use path for main=#1,
    use path for actions=#1,
  }
}
\makeatother

\pgfmathdeclarefunction{sinc}{1}{%
	\pgfmathparse{abs(#1)<0.01 ? int(1) : int(0)}%
	\ifnum\pgfmathresult>0 \pgfmathparse{1}\else\pgfmathparse{sin(#1 r)/#1}\fi%
}
\pgfmathdeclarefunction{gauss}{2}{%
	\pgfmathparse{1/(#2*sqrt(2*pi))*exp(-((x-#1)^2)/(2*#2^2))}%
}

\usepackage[left=2cm,right=2cm,top=2cm,bottom=2cm]{geometry}

%\usepackage[no-math]{fontspec}
%\usepackage{fontspec}
\usepackage{mathspec}
%\usepackage{newtxtext,newtxmath}
%\usepackage{unicode-math}
%\setmainfont{texgyretermes-regular.otf}
%\setsansfont{texgyreheros-regular.otf}
%\newfontfamily\greekfont[Script=Greek]{Linux Libertine O}
%\newfontfamily\greekfontsf[Script=Greek]{Linux Libertine O}
\usepackage{polyglossia}
%\newfontfamily\greekfont[Script=Greek]{texgyretermes-regular.otf}
\newfontfamily\greekfontsf[Script=Greek]{texgyreheros-regular.otf}
\newfontfamily\greekfonttt[Script=Greek]{Latin Modern Mono}
%\usepackage[greek]{babel}
\setdefaultlanguage{greek}
\setotherlanguage{english}

%\usepackage[utf8]{inputenc}
%\usepackage[greek]{babel}


%\usepackage{tkz-euclide} % loads  TikZ and tkz-base
%\usetkzobj{angles} % important you want to use angles

\newlist{enumparen}{enumerate}{1}
\setlist[enumparen]{label=(\arabic*)}
\newlist{enumpar}{enumerate}{1}
\setlist[enumpar]{label=\arabic*)}

\newlist{enumgreek}{enumerate}{1}
\setlist[enumgreek]{label=\alph*.}
\newlist{enumgreekparen}{enumerate}{1}
\setlist[enumgreekparen]{label=(\alph*)}
\newlist{enumgreekpar}{enumerate}{1}
\setlist[enumgreekpar]{label=\alph*)}


\newlist{enumroman}{enumerate}{1}
\setlist[enumroman]{label=(\roman*)}

\newlist{enumlatin}{enumerate}{1}
\setlist[enumlatin]{label=(\alph*)}

\newlist{invitemize}{itemize}{1}
\setlist[invitemize]{noitemsep,label=}

\usepackage{letltxmacro}

\LetLtxMacro\OriginalLongrightarrow\Longrightarrow
\LetLtxMacro\OriginalLongleftarrow\Longleftarrow

% Implement new macros
% --------------------
\usepackage{trimclip}
\DeclareRobustCommand\Longrightarrow{\NewRelbar\joinrel\Rightarrow}
\DeclareRobustCommand\Longleftarrow{\Leftarrow\joinrel\NewRelbar}

\makeatletter
\DeclareRobustCommand\NewRelbar{%
  \mathrel{%
    \mathpalette\@NewRelbar{}%
  }%
}
\newcommand*\@NewRelbar[2]{%
  % #1: math style
  % #2: unused
  \sbox0{$#1=$}%
  \sbox2{$#1\Rightarrow\m@th$}%
  \sbox4{$#1\Leftarrow\m@th$}%
  \clipbox{0pt 0pt \dimexpr(\wd2-.6\wd0) 0pt}{\copy2}%
  \kern-.2\wd0 %
  \clipbox{\dimexpr(\wd4-.6\wd0) 0pt 0pt 0pt}{\copy4}%
}

\def\Rightarrowfill@{\arrowfill@\NewRelbar\NewRelbar\Rightarrow}
\def\Leftarrowfill@{\arrowfill@\Leftarrow\NewRelbar\NewRelbar}

% Fix long xleft(right)arrow
%FIXME: Fails for long arrows
%\patchcmd{\arrowfill@}{-2mu}{-10mu}{}{}
%\patchcmd{\arrowfill@}{-2mu}{-10mu}{}{}
%\patchcmd{\arrowfill@}{-7mu}{-5mu}{}{}
%\patchcmd{\arrowfill@}{-7mu}{-5mu}{}{}
  \patchcmd{\arrowfill@}{-7mu}{-14mu}{}{}
  \patchcmd{\arrowfill@}{-7mu}{-14mu}{}{}
  \patchcmd{\arrowfill@}{-2mu}{-4mu}{}{}
  \patchcmd{\arrowfill@}{-2mu}{-4mu}{}{}
\makeatother


\makeatletter
\pgfdeclareradialshading[tikz@ball]{ball}{\pgfqpoint{0bp}{0bp}}{%
	color(0bp)=(tikz@ball!50!white);
	color(10bp)=(tikz@ball!50!white);
	color(15bp)=(tikz@ball!70!black);
	color(20bp)=(black!70);
	color(30bp)=(black!70)}%
\makeatother


\makeatletter
\let\anw@true\anw@false

%\newcommand{\attnboxed}[1]{\textcolor{red}{\fbox{\normalcolor\m@th$\displaystyle#1$}}}
\makeatother
\tcbset{highlight math style={enhanced,colframe=red,colback=white,%
        arc=0pt,boxrule=1pt,shrink tight,boxsep=1.5mm,extrude by=0.5mm}}
\newcommand{\attnboxed}[1]{\tcbhighmath[colback=red!5!white,drop fuzzy shadow,arc=0mm]{#1}}
\newcommand{\infoboxed}[1]{%
	\tcbhighmath[colframe=blue!50!white,colback=blue!5!white,arc=0mm]{#1}}
\titleformat{\section}{\bf\Large}{Κεφάλαιο \thesection}{1em}{}
\newtcolorbox{attnbox}[1]{colback=red!5!white,%
    colframe=red!75!black,fonttitle=\bfseries,title=#1}
\newtcbox{quickattnbox}[1]{colback=red!5!white,%
	colframe=red!75!black,fonttitle=\bfseries,title=#1}
\newtcolorbox{infobox}[1]{colback=blue!5!white,%
    colframe=blue!75!black,fonttitle=\bfseries,title=#1}

\AtBeginDocument{%
\let\arg\relax
\let\Re\relax
\let\Im\relax
\DeclareMathOperator{\arg}{Arg}
\DeclareMathOperator{\Re}{Re}
\DeclareMathOperator{\Im}{Im}
}
\DeclareMathOperator{\sinc}{sinc}
\DeclareMathOperator{\sgn}{sgn}
\DeclareMathOperator{\erf}{erf}
\DeclareMathOperator{\cov}{cov}

\newif\ifhidetikz
\hidetikzfalse
%\hidetikztrue   % <---- comment/uncomment that line

\ifhidetikz

\let\oldtikzpicture\tikzpicture
\let\oldendtikzpicture\endtikzpicture

\renewenvironment{tikzpicture}{
    \tiny
    \tt
    \color{blue}
    \newcommand{\draw}{\textit{draw}}
    \newcommand{\filldraw}{\textit{filldraw}}
    %\newcommand{\x}{\textit{x}}
    %\newcommand{\p}{\textit{x}}
    \newcommand{\x1}{\textit{x1}}
    \newcommand{\y1}{\textit{y1}}
    \newcommand{\p1}{\textit{p1}}
}{
}
\newenvironment{axis}{
    \newcommand{\addplot}{\textit{addplot}}
}{
}
\fi

\DeclareSIUnit \voltampere { VA } %apparent power 
\DeclareSIUnit \var { VAr } %volt-ampere reactive - idle power 
\DeclareSIUnit \decade { dec } %decade

% Global amount of samples
% Set to a higher value (e.g. 200) for nicer graphs
% Set to a low value (e.g. 10) for performance
\newcommand*{\gsamples}{70}

% Equals command as a workaround for CircuiTikZ bug
% not allowing the = sign in labels
\newcommand*{\equals}{=}

\newcommand{\nesearrow}{%
	\,%
	\smash{\raisebox{-1.1ex}
		{$%
			\stackrel{\displaystyle\nearrow}{\displaystyle\searrow}%
			$}}%
}
\newcommand{\degree}{^{\circ}} % not great
\newcommand\numberthis{\addtocounter{equation}{1}\tag{\theequation}} % add an equation number to a number-less math environment

\newtcbtheorem[number within=section]{theorem}{Θεώρημα}%
{colback=green!5,colframe=green!35!black,colbacktitle=green!35!black,fonttitle=\bfseries,enhanced,attach boxed title to top left={yshift=-2mm,xshift=-7mm},width=.9\textwidth,arc=.7mm}{th}
\newtcbtheorem[number within=section]{defn}{Ορισμός}%
{colback=blue!5,colframe=cyan!35!black,colbacktitle=blue!35!black,fonttitle=\bfseries,enhanced,attach boxed title to top left={yshift=-2mm,xshift=-2mm}}{def}
\newtcbtheorem[number within=section]{exercise}{Άσκηση}%
{colback=gray!3,colframe=gray!35!black,colbacktitle=gray!35!black,fonttitle=\bfseries,enhanced,attach boxed title to top left={yshift=-2mm,xshift=-2mm}}{exc}




\title{Θεωρία Σημάτων και Γραμμικών Συστημάτων - Σημειώσεις}
\date{2016}
\author{\textlatin{\csuse{no\greek @numbers}\selectlanguage{english} \url{https://github.com/kongr45gpen/ece-notes}}}



\begin{document}
	Την Τρίτη μάθημα 8:30 χωρίς διάλειμμα

	\include{signals/chap0}
        \section{Συναρτηστιακοί χώροι}
    Διανυσματικός χώρος \( S \)
    \[
    \bar x,\quad \bar y\quad S
    \]

    \paragraph{Εσωτερικό γινόμενο}
    \[ \left\langle\bar x,\bar y\right\rangle\ \in \mathbb C  \]
    \begin{enumpar}
        \item \( \left\langle\bar x,\bar y\right\rangle
        = \left\langle\bar y,\bar x\right\rangle^* \)
        \item \( c\left\langle\bar x,\bar y\right\rangle
        =\left\langle c\bar x,\bar y\right\rangle \)
        \item \( \left\langle\bar x+\bar y,\bar z\right\rangle
        = \left\langle\bar x,\bar z\right\rangle+\left\langle\bar y,\bar z\right\rangle \)
        \item \( \left\langle\bar x,\bar x\right\rangle \ \geq 0 \) με
        \( \left\langle\bar x,\bar x\right\rangle = 0 \) ανν \( \bar x = \bar 0 \)
    \end{enumpar}

    \paragraph{Νόρμα}
    \[
    \bar x \in S
    \]\[
    ||\bar x|| \geq0
    \]
    \begin{enumpar}
        \item \( ||\bar x|| = 0 \) ανν \( \bar x = \bar 0 \)
        \item \( ||a\bar x|| = |a|||\bar x|| \quad x \in\mathbb C \)
        \item \( ||\bar x+\bar y|| \leq ||\bar x|| + ||\bar y|| \)
    \end{enumpar}
    \paragraph{Μέτρο:} Απόσταση μεταξύ \( \bar x,\bar y \in S \)
    \begin{enumpar}
        \item \( d(\bar x,\bar y)\geq 0 \qquad d(\bar x,\bar y)=0 \)
        ανν \( \bar x = \bar y \)
        \item \( d(\bar x,\bar y) = d(\bar y,\bar x) \)
        \item \( d(\bar x,\bar y) \leq d(\bar x,\bar z) + d(\bar y,\bar z)
        \quad \bar z\in S
         \)

    \end{enumpar}

    \paragraph{Συναρτησιακός χώρος}
    \[
    x(t),y(t) \in S =
    \left\lbrace x(t)/x(t):[t_1,t_2]\to\mathbb R  \right\rbrace
    \]
    \begin{gather*}
    \left\langle
    x(t),y(t)
    \right\rangle  = \int_{t_1}^{t_2}x(t)y(t)\dif t\\
    \left\vert\middle\vert x(t)\middle\vert\right\vert =
    \left[ \int_{t_1}^{t_2} x^2(t)\dif t \right]^{\sfrac{1}{2}}\\
    d\left(
    x(t),y(t)
    \right)=\left[
    \int_{t_1}^{t_2} \left[ x(t)-y(t) \right]^2\dif t
    \right]^{\sfrac{1}{2}}
    \end{gather*}

    \begin{align*}
    \text{Αν } & \left\langle \phi_1(t),\phi_2(t) \right\rangle
    = 0 \quad \phi_1(t) \perp \phi_2(t) \\
    & \left\langle \phi_1(t),\phi_1(t)\right\rangle = 1 \quad
    \phi_1(t) \text{ κανονική}
    \end{align*}

    \paragraph{Τερατοχώρος}

    \begin{tikzpicture}[scale=1,baseline]
        \draw[->] (-0.5,0) -- (2,0) node[below right] {$\hat x$};
        \draw[->] (0,-0.5) -- (0,2) node[above left] {$\hat y$};

        \draw[thick,->] (0,0) -- (1.5,1.5) node[right] {$\vec a$};
        \draw[thick,->] (0,0) -- (1.5,0) node[below] {$\tilde a$};

        \draw (0.2,0) -- (0.2,0.2) -- (0,0.2);
    \end{tikzpicture}

    \( \hat x,\hat y \) όχι εξαρτημένα (συνευθειακά)

    Ποια είναι η καλύτερη προσέγγιση για το \(\vec a\) εφ' όσον δεν υπάρχει
    το \( \vec y \);

    \( \tilde a \) best γιατί \(\mathrm d(\vec a,\tilde a)\) min.

    Άρα:
    \begin{gather*}
        \tilde a = k\hat x \\
        \vec a = a_x\hat x+a_y\hat y\\
        \vec a -\tilde a = (a_x-k)\hat x - a_y\hat y\\
        d(\vec a,\tilde a) = \sqrt{(a_x-k)^2+a_y^2} \\
        \od{}{k}\left( d(\vec a,\tilde a) \right) = \frac{a_x-k}{\cdots}
        = 0 \implies k=a_x = \tilde a \cdot \hat x\\
        \boxed{\vec a\cdot\hat x = a_x}
    \end{gather*}

    Η βέλτιστη έκφραση του \( \vec a \) στο δισδιάστατο χώρο είναι το ίδιο το \( \vec a \).

    \paragraph{Μη κάθετα διανύσματα} \hspace{0pt}

    \begin{tikzpicture}[scale=1]
    \draw[->,dotted] (4,0) -- (6,0);
    \draw[dotted] (6,0) -- (6,2);
    \draw[->] (0,0) -- (4,0) node[below right] {$\hat x$};
    \draw[->] (0,0) -- (6,2) node[right] {$\vec a$};

    \draw[->] (0,0) -- (50:2) node[above right] {$\hat y$};
    \draw[->,blue,thick] (0.5,0) node[below] {$\mathsmaller{<\frac{\pi}{2}}$}
    arc (0:50:0.5);
    \end{tikzpicture}

    \begin{tikzpicture}[scale=1]
    \draw[very thick,draw=blue,->] (0,0) -- (29.8:4.88)
    node[above right] {$\vec a = a_x\hat x + a_y\hat y$};

    \draw[->] (0,0) -- (3,0) node[below right] {$\hat x$};

    \draw[very thick,blue!50!green,yshift=-5pt,xshift=2pt,mark position=1(X),->]
    (0,0) -- (2,0) node[midway,below] {$a_x$};
    \draw[very thick,blue!50!green,yshift=4pt,xshift=-4pt,mark position=1(Y),->]
    (0,0) -- (50:3) node[midway,above,sloped] {$a_y$};

    \draw[->] (0,0) -- (50:2) node[above right] {$\hat y$};

    \draw[dashed] (X) -- ++(50:3.4);
    \draw[dashed] (Y) -- ++(2.5,0);

    \draw (5,0.5) node {$\hat x \nperp \hat y$};
    \end{tikzpicture}

    \begin{gather*}
    \vec a = a_x\hat x+a_y\hat y\\
    \vec a - \tilde a = (a_x-k)\hat x+a_y\hat y\\
    \mathrm d(\vec a,\tilde a)= ||\vec a - \tilde a|| = \sqrt{(\vec a-\tilde a)(\vec a-\tilde a)}
    =\left(
    \left[ (a_x-k)\hat x+a_y\hat y \right]\cdot
    \left[ (a_x-k)\hat x+a_y\hat y \right]
    \right)^{\sfrac{1}{2}}\\
    \left[
    (a_x-k)^2+a_y^2+2(a_x-k)a_y\hat x\cdot\hat y
    \right]^{\sfrac{1}{2}} \\
    \vec a_{\mathrm{best}} = (\vec a \cdot \hat x)\hat x \neq a_x \\
    \boxed{\vec a \cdot \hat x = a_x+a_y\cos\phi\neq a_x}
    \quad \leftarrow
    \text{Η βέλτιστη περιγραφή του $\vec a$ στον διδιάστατο στον άλλο χώρο}
    \end{gather*}

    \paragraph{Συναρτησιακός κόσμος}
    \( \phi_n(t) \) παράγουν χώρο με το μηχανισμό:
    \[
    f(t)=\sum_{n=0}^\infty a_n\phi_n(t)\quad t \in \Delta
    \]

    \( \phi_n(t) \) ανεξάρτητες μεταξύ τους (βάση απειροδιάστατου χώρου)

    \[
    \hat f(t)=\sum_{n=0}^M \underbrace{\hat a_n}_{%
   	\mathclap{\raisebox{-2ex}{\ensuremath{ \scriptstyle
   				\neq a_n\text{, επειδή η βάση δεν είναι ορθοκανονική}
   			}}}
   		}\phi_n(t) \text{ βέλτιστη, ώστε η απόσταση με την $f$
   		να είναι ελάχιστη}
   	\]

    \begin{align*}
        \overbrace{I^2}^{\mathclap{\text{σφάλμα}}} &=
        \int_\Delta\left[ f(t)-\hat f(t) \right]^2\dif t
        \\ &=
        \int_\Delta \left[
            \sum_{n=0}^{+\infty}a_n\phi_n(t)-\sum_{n=0}^M\hat a_n\phi(t)
        \right]^2\dif t
        \\ &= \int_\Delta f^2(t)\dif t + \int_\Delta \left(
            \sum_{n=0}^M \hat a_n\phi_n(t)
        \right)^2\dif t -2\int_{\Delta}\left[
            f(t)\sum_{n=0}^M \hat a_n\phi_n(t)
        \right] \dif t
    \end{align*}

    Άρα:
    {
    \setlength{\mathindent}{0cm}
    \begin{align*}
        I^2 &= \int_{\Delta} f^2(t)\dif t + \int_\Delta \sum_{n=0}^M \left[
            \hat a_n\phi_n(t)
        \right]^2\dif t \\ &\hphantom{=}+ 2 \int_\Delta \left[
            \sum_{n=0}^M\sum_{m=n+1}^M \hat a_n\cdot\hat a_m \phi_n(t)\phi_m(t)
        \right] \dif t
        \\ &\hphantom{=}
        - 2 \int_\Delta \sum_{n=0}^M \hat a_n f(t)\phi_n(t)\dif t
        \\ &= \int_\Delta f^2(t)\dif t + \sum_{n=0}^M \hat a_n^2\int_\Delta
        \phi_n^2(t)\dif t + 2 \sum_{n=0}^M \sum_{n=m+1}^M \hat a_n\hat a_m
        \int_\Delta \phi_n(t)\phi_m(t)\dif t - 2\sum_{n=0}^M \hat a_n
        \int_\Delta f(t)\phi_n(t)\dif t\\
        \od{(I^2)}{\underbrace{\hat a_i}_{\mathclap{\text{από $0$ έως $n$}}}}
        &= 2\hat a_i\int_\Delta \phi_i^2(t)\dif t +2\sum_{m\neq i}
        \hat a_m\int_\Delta \phi_i(t)\phi_m(t)\dif t - 2\int_\Delta f(t)\phi(t)\dif t = 0
    \end{align*}
    }

    \paragraph{Σύστημα εξισώσεων}
    Αν \( \phi_i^{(t)} \) μοναδιαία, τότε: \( \int_\Delta \phi_i^2(t)\dif t = 1 \)

    Αν \( \phi_i(t) \) είναι ορθογώνια, τότε:
    \( \int_{\Delta}  \phi_i(t) \phi_j(t)\dif t = 0, \quad i \neq j \)

    Αν \( \left\lbrace \phi_i(t) \right\rbrace \) είναι ορθοκανονική βάση, τότε:
    \begin{gather*}
    2\vec{a_i}-2\int_{\Delta} f(t)\phi(t)\dif t = 0 \implies
    \vec{a_i} = \overbrace{\int_\Delta
    \underset{\substack{\downarrow\\\mathclap{\raisebox{-1ex}{\text{\scriptsize%
    			προβολή του διανύσματος στο μοναδιαίο}}}}}{f(t)}
    \phi_i(t)\dif t}^{\mathclap{\text{όπως το } k=\vec a \cdot \vec x}} = a_i
    \end{gather*}

    Με άλλη γραφή:
    \[
    2\vec{a_i}\left\langle\phi_i,\phi_i \right\rangle
    +2\sum_{m+i}\vec{a_m}\left\langle \phi_i,\phi_m \right\rangle
    -2\left\langle f,\phi_i \right\rangle=0
    \]

    Είναι:
    \[
    \left\langle f,\phi_i \right\rangle = \left\langle
    \sum_{n=0}^{+\infty}a_n\phi_n,\phi_i\right\rangle
    = \sum_{n=0}^{+\infty} a_n\left\langle \phi_n,\phi_i \right\rangle
    = a_i \quad \text{όπως στα διανύσματα}
    \]

    Ηθικό δίδαγμα: Αν η βάση του χώρου είναι ορθοκανονική και μας ζητηθεί να υπολογίσουμε
    μία προσέγγιση της συνάρτησης σε έναν υποχώρο, μπορούμε άμεσα να υπολογίσουμε την
    προβολή της συνάρτησης πάνω στη βάση.

    \paragraph{Ex.}
    \( f(t)=e^{-3t}\mathrm u(t)
    \qquad \phi_1(t)=e^{-t}\mathrm u(t) \quad \& \quad
    \phi_2(t) = e^{-2t}\mathrm u(t)
     \)

    \begin{gather*}
    \overbrace{\hat f(t)}^{\mathclap{\text{βέλτιστη}}} = a_1e^{-t}\mathrm u(t) +
     a_2e^{-2t}\mathrm u(t) \\
    \int \left[ a_1\phi_1+a_2\phi_2 -f \right]\phi_1\dif t = 0\\
    \int_0^\infty \left[ a_1e^{-t}+a_2e^{-2t}-e^{-3t}\right]e^{-t}\dif t = 0\implies
    \\
    a_1\int_0^\infty e^{-2t}\dif t+a_2\int_0^\infty e^{-3t}\dif t -\int_0^\infty
    e^{-4t}\dif t = 0
    \implies \boxed{ \frac{a_1}{2}+\frac{a_2}{3}-\frac{1}{4} = 0 } \\
    \int\left[ a_1e^{-t}+a_2e^{-2t}-e^{-3t} \right]e^{-2t}\dif t = 0 \implies
    \boxed{\frac{a_1}{3}+\frac{a_2}{4}-\frac{1}{5}=0}\\
    a_1 = -\sfrac{3}{10},\ a_2 = \sfrac{6}{5}
    \end{gather*}

    \paragraph{}
    \[
    \underset{ \text{γενικότερη μορφή}
    	}{\boxed{E \overset{\triangle}{=} \int_\Delta f^2(t)\dif t
    = \sum_{n=0}^{\infty} a_n^2 \quad
    \begin{array}{l} \text{Parseval's} \\ \text{Theorem}\end{array}}}
    \]


    \section{Ανάλυση Fourier}
	\subsubsection{Περιοδικές συναρτήσεις με περίοδο $T$}
	\begin{gather*}  
	x_k =\sqrt{\frac{2}{T}} \cos(k\omega t)\qquad \omega = \frac{2\pi}{T}
	\text{ θεμελιώδης κυκλική συχνότητα}
	\\
	\left\langle 
	x_k(t),x_n(t)
	\right\rangle =
	\int_{-\sfrac{T}{2}}^{\sfrac{T}{2}} x_k(t)x_n(t)\dif t
	= \int_{-\sfrac{T}{2}}^{\sfrac{T}{2}} \cos(k\omega t)\cos(n\omega t)\dif t
	= \begin{cases}
	n \neq k \to 0 \\ n = k \to 1
	\end{cases}
	\end{gather*}
	\begin{gather*}
		y_k = \sqrt{\frac{2}{T}}\sin(k\omega t)
		\left\langle
		y_k,y_n
		\right\rangle = \begin{cases}
		n \neq k \to 0 \\ n = k \to 1
		\end{cases}
	\end{gather*}
	
	Υποστηρίζω ότι κάθε περιοδική \( \displaystyle 
	f(t) = \sum_{n=0}^\infty a_n\cos(n\omega t)+b_n\sin(n\omega t) \)
	
	Οραματίζομαι ότι αν η παραπάνω \( f(t) \) είναι σήμα εισόδου σε ένα σύστημα,
	τα ημίτονα και συνημίτονα ως ιδιοσυναρτήσεις θα παραμείνουν αμετάβλητα,
	και θα τροποποιηθούν μόνο τα \( a_n, b_n \).
	
	\begin{gather*}
	z_k(t) =  e^{jk\omega t} \\
	\left\langle
	z_k,z_n
	\right\rangle = \begin{cases}
	k\neq n \to 0 \\
	k = n \to T
	\end{cases}\\
	z_k(t)=\frac{1}{\sqrt{T}}e^{jk\omega t}
	\end{gather*}
	
	\begin{gather*}
	\boxed{
	f(t) = \sum_{n=-\infty}^\infty F_n e^{jn\omega t}
    } \text{ εκθετική σειρά} \\
    \boxed{
    	f(t) =\frac{a_0}{2} + \sum_{n=1}^\infty \left[
    	a_n\cos(n\omega t)+b_n\sin(n\omega t)
    	\right]
    	} \text{ τριγωνομετρική σειρά Α} \\
    \boxed {
    	f(t) = A_0 + \sum_{n=1}^\infty A_n\cos(n\omega t+\phi_n)
    	} \text{ τριγωνομετρική σειρά Β}
	\end{gather*}
	
	Οι συντελεστές μπορούν να βρεθούν από τις προβολές της συνάρτησης πάνω στα
	ημίτονα και τα συνημίτονα:
	\begin{align*}
	a_n &= \frac{2}{T} \int_{-\sfrac{T}{2}}^{\sfrac{T}{2}} f(t)\cos(n\omega t)\dif t
	\quad n \neq 0 \\
	b_n &= \frac{2}{T} \int_{-\sfrac{T}{2}}^{\sfrac{T}{2}} f(t)\sin(n\omega t)\dif t
	\\ a_0 &= \frac{2}{T} \int_{-\sfrac{T}{2}}^{\sfrac{T}{2}} f(t)\dif t
	\\[5pt] F_n &= \frac{1}{T} \int_{-\sfrac{T}{2}}^{\sfrac{T}{2}}
	f(t)e^{-jn\omega t}\dif t
	\end{align*}
	
	\paragraph{Συνθήκες Dirichlet}
	\begin{enumpar}
		\item \( 
		\displaystyle \int_{-\sfrac{T}{2} }^{\sfrac{T}{2} }\left|
		f(t)
		\right| \dif t < \infty
		 \)
		\item Πεπερασμένο πλήθος ασυνεχειών εντός \( T \)
		\item Πεπερασμένος αριθμός τοπικών ακροτάτων εντός \( T \)
	\end{enumpar}
	
	\( f(t) \) περιοδική \( T \)
	
	\hspace{-2cm}
	\begin{tabular}{|c|c|c|c|}
		\hline \textbf{Μορφή} & \textbf{Σειρά}  & \textbf{Συντελεστές} & \textbf{Αλλαγές} \\ 
		\hline Εκθετική & \(\displaystyle f(t)=\sum_{-\infty}^\infty F_n e^{j\omega n t} \) &
		\(\displaystyle F_n = \frac{1}{T} \int_{-\sfrac{T}{2}}^{\sfrac{T}{2}}
		f(t)e^{-jn\omega t}\dif t\) &\(
		\begin{array}{l}
			F_0 = \sfrac{a_0}{2} \\ F_n = \sfrac{1}{2} (a_n-jb_n)  
		\end{array}\)
		 \\ 
		\hline {\small Τριγωνομετρική} Α  & \( 
		\displaystyle f(t)=\frac{a_0}{2}+\sum_{n=1}^\infty a_n\cos(n\omega t)
		+b_n\sin(n\omega t)
		 \)  & \(
		 \begin{array}{ll}
		 a_n &= \frac{2}{T} \int_{-\sfrac{T}{2}}^{\sfrac{T}{2}} f(t)\cos(n\omega t)\dif t \\
		 b_n &= \frac{2}{T} \int_{-\sfrac{T}{2}}^{\sfrac{T}{2}} f(t)\sin(n\omega t)\dif t \\ a_0 &= \frac{2}{T} \int_{-\sfrac{T}{2}}^{\sfrac{T}{2}} f(t)\dif t
		 \end{array}
		 \) & 
		 \(
		 \begin{array}{l}
		 a_n = (F_n+F_{-n})\\
		 b_n=j(F_n-F_{-n})\\
		 a_0=2F_0
		 \end{array}
		 \)
		 \\ 
		\hline 
		{\small Τριγωνομετρική} Β
		& \(
		f(t) = A_0 + \sum_{n=1}^\infty A_n\cos(n\omega t+\phi_n)
		\) & &
		\(
		\begin{array}{l}
		A_0 = \sfrac{a_0}{2} \\
		A_n = \sqrt{a_n^2+b_n^2} = 2|F_n| \\
		\phi_n = \arctan\left(\frac{b_n}{a_n}\right)
		\end{array}
		\)
		\\ \hline
	\end{tabular} 
	
	\paragraph{}
	\begin{align*}
	P &= \frac{W}{T} = \frac{1}{T} \int_0^T f^2(t)\dif t =	
	\frac{a_0^2}{4}+\frac{1}{2}\sum_{n=1}^\infty a_n^2+b_n^2
	\\ &= F_0^2+2\sum_{n=1}^\infty |F_n|^2 = 
	\sum_{n=-\infty}^\infty \left|F_n\right|^2 
	\end{align*}
	
	\paragraph{Άσκηση για το σπίτι}
	Να βρεθούν η εκθετική και η τριγωνομετρική σειρά του σήματος:
	%TODO Rekanos Graph 13
	%TODO Rekanos Graph 14
	
	\subsection{Μετασχηματισμός Fourier}
	%TODO Rekanos Graph 15
	\paragraph{Φορέας} συνάρτησης είναι το διάστημα του πεδίου ορισμού της στο
	οποίο η συνάρτηση δεν είναι 0 (από \( \min x \) για το οποίο δεν είναι 0 ως
	το αντίστοιχο \( \max x \)).
	
	\begin{gather*}
	\tilde f(t) = \sum{k=-\infty}^\infty f(t-kT)\\
	\downarrow T\text{-περιοδική} \rightarrow \tilde f(t)=\frac{a_0}{2}
	+\sum_{n=1}^\infty a_n\cos(\omega_0 n t)+b_n\sin(\omega_0 n t)
	\qquad \omega_0=\frac{2\pi}{T} \\[3pt]
	f(t)=\begin{cases}
	\tilde f(t) &\quad -\sfrac{T}{2}\leq t \leq \sfrac{T}{2}\\
	0 &\quad\text{elsewhere} 
	\end{cases}
	\end{gather*}
	
	%TODO Rekanos Graph 16
	
	\paragraph{Μετασχηματισμός Fourier}
	\begin{align*}
	F(\omega) &\overset{\triangle}{=}
	\int_{-\infty}^{\infty} f(t)e^{-j\omega t}\dif t
	\\ f(t) &= \frac{1}{2\pi}\int_{-\infty}^{\infty}
	F(\omega )e^{j\omega t}\dif\omega
	\end{align*}
	
	\begin{attnbox}{Προσοχή}
		Όταν παίρνουμε τύπους από τυπολόγια, ελέγχουμε τον ορισμό
		του μετασχηματισμού Fourier, για διαφορές στη σύμβαση!
	\end{attnbox}
	
	\paragraph{Αντίστοιχος ορισμός}
	\begin{align*}
	F(\mathfrak{f}) &\overset{\triangle}{=} \int_{-\infty}^{\infty}
	e^{-j2\pi\mathfrak{f}t}\dif t\\
	f(t) &= \int_{-\infty}^{\infty} F(\mathfrak f)e^{j2\pi\mathfrak{f}t}\dif
	\mathscr{f}
	\end{align*}
	(όπου \( \mathfrak{x} \) η συχνότητα)
	
	Η αρνητική συχνότητα δεν έχει καμία φυσική σημασία!
	 
	 \subsubsection{Ιδιότητες}
	 \begin{itemize}

     \item
	 \begin{gather*}
	 F(\omega)= A(\omega )e^{j\Phi(\omega )} =
	 \underbrace{F_R(\omega )}_{\mathclap{\Re\left\lbrace F(\omega ) \right\rbrace}}
	 +j\underbrace{F_i(\omega )}_{\mathclap{\Im\left\lbrace F(\omega ) \right\rbrace}}
	 \\ A(\omega )=\left| F(\omega ) \right|
	 \end{gather*}
	 \item
	 \begin{gather*}
	 F(\omega )=\int_{-\infty}^{\infty} f(t)\left(\cos(\omega t)-j\sin(\omega t)\right)
	 \dif t = \underbrace{\int_{-\infty}^{\infty} f(t)\cos\omega t\dif t}_{%
	 	\Re\left\lbrace F(\omega ) \right\rbrace
	 	}
	 	\underbrace{-}_{-}
	 	j \underbrace{\int_{-\infty}^{\infty} f(t)\sin\omega t\dif t}_{%
	 		\Im\left\lbrace F(\omega ) \right\rbrace
	 		}
	 \end{gather*}
	 
	 Αν \( f(t):\mathbb R \to\mathbb R  \) είναι άρτια\\
	 \( F(\omega ) \) είναι πραγματική \quad \( 
	 F(\omega ) \equiv \Re\left\lbrace F(\omega ) \right\rbrace
	  \) και είναι άρτια
	  
	 Αν \( f(t):\mathbb R \to\mathbb R  \) είναι περιττή\\
	 \( F(\omega ) \) είναι φανταστική \quad \( 
	 F(\omega ) = j\Im \left\lbrace F(\omega) \right\rbrace
	  \) και είναι περιττή
	  
	 Κάθε συνάρτηση είναι άθροισμα μίας άρτιας και μίας περιττής. Έστω
	 \( f(t) = f_e(t)+f_o(t) \). Τότε:
	 \begin{align*}
	 F(\omega ) &= \int_{-\infty}^{\infty} \left(
	 f_o(t)+f_e(t)
	 \right)(\cos\omega t -j\sin \omega t)\dif t
	 \\ &= \cancel{\int_{-\infty}^{\infty} f_o\cos\omega t\dif t}
	 + \int_{-\infty}^{\infty} f_e\cos\omega t\dif t
	 - j\int_{-\infty}^{\infty} \sin\omega t\dif t
	 - j \cancel{\int_{-\infty}^{\infty} f_e\sin\omega t\dif t }
	 \end{align*}
	 
	 Αν η \( f \) είναι πραγματική: \\
	 \( \Re\left\lbrace F(\omega ) \right\rbrace \) είναι άρτια \\
	 \( \Im\left\lbrace F(\omega ) \right\rbrace \) είναι περιττή \\
	 \( A(\omega)=\left|F(\omega )\right| = 
	 \sqrt{\Re^2\left\lbrace F(\omega ) \right\rbrace}
	 +\Im^2\left\lbrace F(\omega) \right\rbrace
	  \) είναι άρτια\\
	  \( \Phi(\omega) =\arctan
	  \frac{\Im\left\lbrace F(\omega) \right\rbrace}%
	  {\Re\left\lbrace F(\omega ) \right\rbrace}
	   \) είναι περιττή.
	   
	 Αν η \( f(t):\mathbb R \to \mathbb R  \) και άρτια:
	 \begin{itemize}
	 	\item \( \Im\left\lbrace F(\omega)=0 \right\rbrace \)
	 	\item \( \Phi(\omega) = 0 \)
	 \end{itemize}
	 
	 Αν η \( f(t):\mathbb R\to\mathbb R  \) είναι περιττή:
	 \begin{itemize}
	 	\item \( \Re\left\lbrace F(\omega) \right\rbrace = 0 \)
	 \end{itemize}
	 
	 \item
	 
	 Αν \( f_1(t) \xrightarrow{\text{FT}} F_1(\omega ) \) και
	 \( f_2(t)\xrightarrow{\text{FT}} F_2(\omega ) \) \\
	 \( \forall a_1,a_2\in\mathbb C \) σταθερά: \\
	 \( f(t) = a_1f_1(t)+a_2f_2(t) \xrightarrow{\text{FT}}
	 F(\omega) = a_1F_1(\omega)+a_2F_2(\omega)
	  \) \\
	  Γραμμικότητα του Fourier Transform
	  
	 \item Συμμετρική ιδιότητα (το διπλάσιο τυπολόγιο)
	 
	 Αν \( f(t)\xrightarrow{\text{FT}} F(\omega)
	 \qquad F(t)\xrightarrow{\text{FT}} 2\pi f(-\omega)
	  \)
	  
	 \item 
	 \begin{align*}
	 f(t) &\to F(\omega ) \quad = A(\omega) e^{j\Phi(\omega )} \\
	 f(t-\tau) &\to e^{-j\omega \tau}F(\omega ) \quad =
	 A(\omega )e^{j\left( \Phi(\omega )-\omega \tau \right)}
	 \end{align*}
	 
	 \item
	 \begin{align*}
	 f(t)&\to F(\omega) \\
	 e^{j\omega_0 t}f(t) &\to F(\omega-\omega_0) \\[7pt]
	 \text{π.χ}\quad \cos(\omega_0 t)f(t)=
	 \frac{e^{j\omega_0 t}+e^{-j\omega_0 t}}{2}f(t) &\xrightarrow{\text{FT}}
	 \frac{1}{2}\left[
	 F(\omega-\omega_0)+F(\omega+\omega_0)
	 \right]
	 \end{align*}
	 
	 \end{itemize}
	 
	 \paragraph{Κλιμάκωση στο χρόνο}
	 \begin{align*}
	 f(t) &\to F(\omega )\\
	 f(at)&\to \frac{1}{|a|}F\left( \frac{\omega }{a} \right)
	 \quad \text{γιατί; να αποδειχθεί στο σπίτι!}
	 \end{align*}
	 
     \paragraph{Τι συμβαίνει με τη συνέλιξη}
     \begin{align*}
     g(t) &= x(t)*h(t) \\
     x(t)&\to X(\omega )\\
     h(t)&\to H(\omega )\\
     y(t)&\to Y(\omega) = X(\omega)H(\omega)
     \end{align*}
     
     \begin{align*}
     y(t) &= x(t)\cdot h(t)\\
     y(t) &\to Y(\omega) = \frac{1}{2\pi}X(\omega)*H(\omega )
     \\ &= \frac{1}{2\pi} \int_{-\infty}^{\infty} X(\xi)H(\omega-\xi)\dif\xi
     \end{align*}
     
    
\end{document}