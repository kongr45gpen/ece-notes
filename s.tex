% !TeX program = xelatex
\documentclass[11pt,a4paper,notitlepage,fleqn]{article}

\usepackage{amsmath}
\usepackage{amsfonts}
\usepackage{amssymb}
\usepackage{libs/commath2}
\usepackage[table]{xcolor}
\usepackage[hidelinks,draft=false]{hyperref}
\usepackage[skins,theorems]{tcolorbox}
\usepackage{titlesec}
\usepackage{tikz}
\usepackage{libs/circuitikz} % use our own recent version to make sure some bugs are fixed
\usepackage{pgfplots}
\usepackage{mathtools}
\usepackage[makeroom]{cancel}
\usepackage{mathrsfs}
\usepackage{wrapfig}
%\usepackage{subcaption}
%\usepackage{floatrow}
\usepackage{esint}
\usepackage{enumitem}
%\usepackage{bm}
\usepackage{relsize}
\usepackage{xfrac}
\usepackage{comment}
\usepackage{siunitx}
%\usepackage{MnSymbol}
\usepackage[obeyDraft,disable]{todonotes}
%\usepackage[linesnumbered,lined]{algorithm2e}


\pgfplotsset{compat=1.13}
\usetikzlibrary{arrows.meta}
\usetikzlibrary{patterns}
\usetikzlibrary{decorations.pathmorphing,patterns}
\usetikzlibrary{decorations.markings}
\usetikzlibrary{backgrounds}
\usetikzlibrary{shapes.misc}
\usetikzlibrary{shapes.multipart}
\usetikzlibrary{shadows.blur}
\usetikzlibrary{fadings}
\usetikzlibrary{intersections}
\usetikzlibrary{arrows.meta}
\usetikzlibrary{calc}
\usetikzlibrary{matrix}
\usetikzlibrary{positioning}
\usetikzlibrary{shapes}
\usetikzlibrary{shadings}

\tcbuselibrary{breakable}

\tikzset{cross/.style={cross out, draw,
        minimum size=2*(#1-\pgflinewidth),
        inner sep=0pt, outer sep=0pt}}
\tikzset{
    mark position/.style args={#1(#2)}{
        postaction={
            decorate,
            decoration={
            	post length=1mm, % ??? Magic to fix "Dimension
            	pre length=1mm, % ???  too large" errors.
                markings,
                mark=at position #1 with \coordinate (#2);
            }
        }
    }
}
\makeatletter
\tikzset{
  use path for main/.code={%
    \tikz@addmode{%
      \expandafter\pgfsyssoftpath@setcurrentpath\csname tikz@intersect@path@name@#1\endcsname
    }%
  },
  use path for actions/.code={%
    \expandafter\def\expandafter\tikz@preactions\expandafter{\tikz@preactions\expandafter\let\expandafter\tikz@actions@path\csname tikz@intersect@path@name@#1\endcsname}%
  },
  use path/.style={%
    use path for main=#1,
    use path for actions=#1,
  }
}
\makeatother

\pgfmathdeclarefunction{sinc}{1}{%
	\pgfmathparse{abs(#1)<0.01 ? int(1) : int(0)}%
	\ifnum\pgfmathresult>0 \pgfmathparse{1}\else\pgfmathparse{sin(#1 r)/#1}\fi%
}
\pgfmathdeclarefunction{gauss}{2}{%
	\pgfmathparse{1/(#2*sqrt(2*pi))*exp(-((x-#1)^2)/(2*#2^2))}%
}

\usepackage[left=2cm,right=2cm,top=2cm,bottom=2cm]{geometry}

%\usepackage[no-math]{fontspec}
%\usepackage{fontspec}
\usepackage{mathspec}
%\usepackage{newtxtext,newtxmath}
%\usepackage{unicode-math}
%\setmainfont{texgyretermes-regular.otf}
%\setsansfont{texgyreheros-regular.otf}
%\newfontfamily\greekfont[Script=Greek]{Linux Libertine O}
%\newfontfamily\greekfontsf[Script=Greek]{Linux Libertine O}
\usepackage{polyglossia}
%\newfontfamily\greekfont[Script=Greek]{texgyretermes-regular.otf}
\newfontfamily\greekfontsf[Script=Greek]{texgyreheros-regular.otf}
\newfontfamily\greekfonttt[Script=Greek]{Latin Modern Mono}
%\usepackage[greek]{babel}
\setdefaultlanguage{greek}
\setotherlanguage{english}

%\usepackage[utf8]{inputenc}
%\usepackage[greek]{babel}


%\usepackage{tkz-euclide} % loads  TikZ and tkz-base
%\usetkzobj{angles} % important you want to use angles

\newlist{enumparen}{enumerate}{1}
\setlist[enumparen]{label=(\arabic*)}
\newlist{enumpar}{enumerate}{1}
\setlist[enumpar]{label=\arabic*)}

\newlist{enumgreek}{enumerate}{1}
\setlist[enumgreek]{label=\alph*.}
\newlist{enumgreekparen}{enumerate}{1}
\setlist[enumgreekparen]{label=(\alph*)}
\newlist{enumgreekpar}{enumerate}{1}
\setlist[enumgreekpar]{label=\alph*)}


\newlist{enumroman}{enumerate}{1}
\setlist[enumroman]{label=(\roman*)}

\newlist{enumlatin}{enumerate}{1}
\setlist[enumlatin]{label=(\alph*)}

\newlist{invitemize}{itemize}{1}
\setlist[invitemize]{noitemsep,label=}

\usepackage{letltxmacro}

\LetLtxMacro\OriginalLongrightarrow\Longrightarrow
\LetLtxMacro\OriginalLongleftarrow\Longleftarrow

% Implement new macros
% --------------------
\usepackage{trimclip}
\DeclareRobustCommand\Longrightarrow{\NewRelbar\joinrel\Rightarrow}
\DeclareRobustCommand\Longleftarrow{\Leftarrow\joinrel\NewRelbar}

\makeatletter
\DeclareRobustCommand\NewRelbar{%
  \mathrel{%
    \mathpalette\@NewRelbar{}%
  }%
}
\newcommand*\@NewRelbar[2]{%
  % #1: math style
  % #2: unused
  \sbox0{$#1=$}%
  \sbox2{$#1\Rightarrow\m@th$}%
  \sbox4{$#1\Leftarrow\m@th$}%
  \clipbox{0pt 0pt \dimexpr(\wd2-.6\wd0) 0pt}{\copy2}%
  \kern-.2\wd0 %
  \clipbox{\dimexpr(\wd4-.6\wd0) 0pt 0pt 0pt}{\copy4}%
}

\def\Rightarrowfill@{\arrowfill@\NewRelbar\NewRelbar\Rightarrow}
\def\Leftarrowfill@{\arrowfill@\Leftarrow\NewRelbar\NewRelbar}

% Fix long xleft(right)arrow
%FIXME: Fails for long arrows
%\patchcmd{\arrowfill@}{-2mu}{-10mu}{}{}
%\patchcmd{\arrowfill@}{-2mu}{-10mu}{}{}
%\patchcmd{\arrowfill@}{-7mu}{-5mu}{}{}
%\patchcmd{\arrowfill@}{-7mu}{-5mu}{}{}
  \patchcmd{\arrowfill@}{-7mu}{-14mu}{}{}
  \patchcmd{\arrowfill@}{-7mu}{-14mu}{}{}
  \patchcmd{\arrowfill@}{-2mu}{-4mu}{}{}
  \patchcmd{\arrowfill@}{-2mu}{-4mu}{}{}
\makeatother


\makeatletter
\pgfdeclareradialshading[tikz@ball]{ball}{\pgfqpoint{0bp}{0bp}}{%
	color(0bp)=(tikz@ball!50!white);
	color(10bp)=(tikz@ball!50!white);
	color(15bp)=(tikz@ball!70!black);
	color(20bp)=(black!70);
	color(30bp)=(black!70)}%
\makeatother


\makeatletter
\let\anw@true\anw@false

%\newcommand{\attnboxed}[1]{\textcolor{red}{\fbox{\normalcolor\m@th$\displaystyle#1$}}}
\makeatother
\tcbset{highlight math style={enhanced,colframe=red,colback=white,%
        arc=0pt,boxrule=1pt,shrink tight,boxsep=1.5mm,extrude by=0.5mm}}
\newcommand{\attnboxed}[1]{\tcbhighmath[colback=red!5!white,drop fuzzy shadow,arc=0mm]{#1}}
\newcommand{\infoboxed}[1]{%
	\tcbhighmath[colframe=blue!50!white,colback=blue!5!white,arc=0mm]{#1}}
\titleformat{\section}{\bf\Large}{Κεφάλαιο \thesection}{1em}{}
\newtcolorbox{attnbox}[1]{colback=red!5!white,%
    colframe=red!75!black,fonttitle=\bfseries,title=#1}
\newtcbox{quickattnbox}[1]{colback=red!5!white,%
	colframe=red!75!black,fonttitle=\bfseries,title=#1}
\newtcolorbox{infobox}[1]{colback=blue!5!white,%
    colframe=blue!75!black,fonttitle=\bfseries,title=#1}

\AtBeginDocument{%
\let\arg\relax
\let\Re\relax
\let\Im\relax
\DeclareMathOperator{\arg}{Arg}
\DeclareMathOperator{\Re}{Re}
\DeclareMathOperator{\Im}{Im}
}
\DeclareMathOperator{\sinc}{sinc}
\DeclareMathOperator{\sgn}{sgn}
\DeclareMathOperator{\erf}{erf}
\DeclareMathOperator{\cov}{cov}

\newif\ifhidetikz
\hidetikzfalse
%\hidetikztrue   % <---- comment/uncomment that line

\ifhidetikz

\let\oldtikzpicture\tikzpicture
\let\oldendtikzpicture\endtikzpicture

\renewenvironment{tikzpicture}{
    \tiny
    \tt
    \color{blue}
    \newcommand{\draw}{\textit{draw}}
    \newcommand{\filldraw}{\textit{filldraw}}
    %\newcommand{\x}{\textit{x}}
    %\newcommand{\p}{\textit{x}}
    \newcommand{\x1}{\textit{x1}}
    \newcommand{\y1}{\textit{y1}}
    \newcommand{\p1}{\textit{p1}}
}{
}
\newenvironment{axis}{
    \newcommand{\addplot}{\textit{addplot}}
}{
}
\fi

\DeclareSIUnit \voltampere { VA } %apparent power 
\DeclareSIUnit \var { VAr } %volt-ampere reactive - idle power 
\DeclareSIUnit \decade { dec } %decade

% Global amount of samples
% Set to a higher value (e.g. 200) for nicer graphs
% Set to a low value (e.g. 10) for performance
\newcommand*{\gsamples}{70}

% Equals command as a workaround for CircuiTikZ bug
% not allowing the = sign in labels
\newcommand*{\equals}{=}

\newcommand{\nesearrow}{%
	\,%
	\smash{\raisebox{-1.1ex}
		{$%
			\stackrel{\displaystyle\nearrow}{\displaystyle\searrow}%
			$}}%
}
\newcommand{\degree}{^{\circ}} % not great
\newcommand\numberthis{\addtocounter{equation}{1}\tag{\theequation}} % add an equation number to a number-less math environment

\newtcbtheorem[number within=section]{theorem}{Θεώρημα}%
{colback=green!5,colframe=green!35!black,colbacktitle=green!35!black,fonttitle=\bfseries,enhanced,attach boxed title to top left={yshift=-2mm,xshift=-7mm},width=.9\textwidth,arc=.7mm}{th}
\newtcbtheorem[number within=section]{defn}{Ορισμός}%
{colback=blue!5,colframe=cyan!35!black,colbacktitle=blue!35!black,fonttitle=\bfseries,enhanced,attach boxed title to top left={yshift=-2mm,xshift=-2mm}}{def}
\newtcbtheorem[number within=section]{exercise}{Άσκηση}%
{colback=gray!3,colframe=gray!35!black,colbacktitle=gray!35!black,fonttitle=\bfseries,enhanced,attach boxed title to top left={yshift=-2mm,xshift=-2mm}}{exc}




\usepackage{endnotes}
\usepackage{hyperref}
\usepackage{graphicx}
\usepackage{amsthm}
\usepackage{amssymb}
\usepackage{float}
\usepackage{listings}
\usepackage{minted}

\newtheorem{thm}{Θεώρημα}[section]
\newtheorem{lem}[thm]{Λήμμα}
\newtheorem{cor}[thm]{Πόρισμα}

\title{ΣΑΜΥΥΥΚ
	\\
	{ 
		%\normalsize
		Στοχαστικές Ανελίξεις σε Μη-Υψιπερατά Υπερβολικά Υπερ-Κυματίδια
		\\
		\normalsize Σημειώσεις από τις παραδόσεις}
	}
\date{Φθινόπωρο 2017
	\\
	{ 
	%\small Τελευταία ενημέρωση: \today
	}
}
\author{
	Για τον κώδικα σε \LaTeX, ενημερώσεις και προτάσεις:
	\\
	\url{https://github.com/kongr45gpen/ece-notes}}

\setallmainfonts(Digits,Latin,Greek){Asana Math}
\setmainfont{Noto Serif}
\setsansfont{Ubuntu}
\usepackage{polyglossia}
\newfontfamily\greekfont[Script=Greek,Scale=1.00]{Liberation Serif}

\hypersetup{pdftitle = {Στοχαστικές Ανελίξεις σε Μη-Υψιπερατά μπλα μπλα μπλα μπλα}}

\let\mytodo\todo
\renewcommand{\todo}[1]{\par\mytodo[inline,noline]{#1}}


\begin{document}
\maketitle

\hrule
\vspace{50pt}

\begin{infobox}{Λάθη \& Διορθώσεις}
	Οι τελευταίες εκδόσεις των σημειώσεων βρίσκονται στο Github
	(\url{https://github.com/kongr45gpen/ece-notes/raw/master/s.pdf}) ή
	στη διεύθυνση \url{http://helit.org/ece-notes/s.pdf}.
	
	Περιέχουν διορθώσεις σε λάθη και τυχόν βελτιώσεις.
	
	\tcblower
	
	Μπορείτε να ενημερώνετε για οποιοδήποτε λάθος και πρόταση
	μέσω PM στο forum, issue στο Github, ή οποιουδήποτε άλλου τρόπου.
\end{infobox}

{
	\hypersetup{linkcolor=black}
	\tableofcontents
}

\section{Εισαγωγή}
\label{sec:1}
Καλώς ορίσατε στο μάθημα \textit{Στοχαστικές Ανελίξεις σε Μη-Υψιπερατά Υπερβολικά Υπερ-Κυματίδια}! Το μάθημα αυτό είναι το \textbf{πιο απλό μάθημα της σχολής}. Οι έννοιες είναι προφανείς και εύκολες στην κατανόηση, ειδικά αν παρακολουθείτε στη θεωρία. Τα μαθηματικά βασίζονται σε \textbf{μαθηματικά νηπιαγωγείου}, και όλες οι έννοιες είναι ήδη γνωστές από τα προηγούμενα χρόνια. Το μάθημα είναι τόσο απλό, που μπορείς να το περάσεις με \textbf{διάβασμα 5 λεπτών}, και να πάρεις 10 με \textbf{διάβασμα 1 ώρας}! Παρ' όλα αυτά, ο Μέσος Όρος των φοιτητών τα προηγούμενα χρόνια ήταν 2.33, και κανείς δεν έχει περάσει ποτέ το μάθημα.
\begin{theorem}{}{}
	\large
	Κανείς δεν διαβάζει παραπάνω από 5 λεπτά για το μάθημα \textit{Στοχαστικές Ανελίξεις σε Μη-Υψιπερατά Υπερβολικά Υπερ-Κυματίδια}.
\end{theorem}

\begin{proof}
	Όπως αναφέρθηκε στο \autoref{sec:1}, το μάθημα απαιτεί διάβασμα 5 λεπτών. Έστω ότι αυτό είναι η πρόταση \( A \).
	
	Όμως κανείς δεν έχει περάσει το μάθημα, έστω πρόταση \(B\).
	
	Έστω πρόταση \( Γ\) ότι κανείς δεν διαβάζει παραπάνω από 5 λεπτά το μάθημα.
	
	Τότε έχουμε:
	
	\[
	A \wedge B \implies Γ
	\]
\end{proof}

Το μάθημα βαθμολογείται με εξετάσεις.

Υπάρχουν μόνο τελικές εξετάσεις.


\section{Αβελιανός ισομορφισμός αεροπλάνων}

Αεροπλάνα πάνω από τα τετραγωνικά στοιχεία
Α. Επίθετο
\subsection{Αφηρημένη}

Αφήνω το U ' να είναι μια κενή άλγεβρα. Η κατασκευή των στοιχείων του Grassmann από τον P. Raman ήταν ορόσημο στη σύγχρονη θεωρία μοντέλων. Δείχνουμε ότι κάθε σχεδόν ελλειπτική, εξαιρετικά ανοικτή κατηγορία είναι γενικά p- adic. Το πρόσφατο ενδιαφέρον για απλά ολοκληρώσιμα, ασήμαντα κενά ιδεώδη Peano έχει επικεντρωθεί στην κατασκευή ολοκληρωμένων πολλαπλών εφαρμογών. Στο [ ? ], το κύριο αποτέλεσμα ήταν ο χαρακτηρισμός των γενικώς υποδεκτών υποσυνόλων.

\subsection{Τοπολογίες}

Ο σκοπός του παρόντος άρθρου είναι να χαρακτηρίσει τους τοπολογίες. Συνεπώς, στο πλαίσιο αυτό, τα αποτελέσματα του [ ? ] έχουν μεγάλη σημασία. Δυστυχώς, δεν μπορούμε να υποθέσουμε ότι E ≤ -∞ . Μια χρήσιμη επισκόπηση του θέματος μπορεί να βρεθεί στο [ ? ]. Μια χρήσιμη επισκόπηση του θέματος μπορεί να βρεθεί στο [ ? ]. Κάθε μαθητής έχει επίγνωση ότι το Ω είναι diffeomorphic στο w N. Ένα κεντρικό πρόβλημα στη σύνθετη θεωρία χειριστών είναι ο υπολογισμός υπερβολικών καμπυλών.

Στο [ ? ; ], οι συγγραφείς ασχολούνται με την ύπαρξη εντελώς εγχυτικών αεροπλάνων υπό την επιπρόσθετη υπόθεση ότι υπάρχει ένας συμπαγής γραμμικός θετικός οριστικός, συνδυαστικός φυσικός ομοιομορφισμός. Σε μελλοντικές εργασίες, σχεδιάζουμε να αντιμετωπίσουμε ζητήματα ατιμωρησίας καθώς και μοναδικότητας. Τώρα πρόσφατα, υπήρξε μεγάλο ενδιαφέρον για τον χαρακτηρισμό σχεδόν συμπαγών, ασυνήθιστα μηδενικών, αντικειμενικών τάξεων Grassmann. Τώρα στο μέλλον, σχεδιάζουμε να αντιμετωπίσουμε τα ζητήματα του πεπερασμένου καθώς και της ύπαρξης. Επιπλέον, αυτό αφήνει ανοιχτό το ζήτημα της μοναδικότητας. Δυστυχώς, δεν μπορούμε να το υποθέσουμε αυτό
~𝔭 - 1 () 1 - ı 0	→ ∮ ⋂ R ∈ ~γ tanh (- π) dλ ∧ ⋅⋅⋅ + | Χ U , k | + i 		
= ∫ ∫ inf -1- ∞ d u α, v ∧ q Ω () Ν 0, Δ -6		
≡ Π Ο ∈ v 1- ∅ ⋅ ⋅⋅⋅ ∩ φ(i ∪ wz, γ, - 1) . 		

Στο [ ? ], το κύριο αποτέλεσμα ήταν ο υπολογισμός υπερ-πενιχρών κλιμάκων. Είναι δυνατόν να περιγραφούν άνευ όρων γραμμές Volterra; Εδώ, η αναστρεψιμότητα είναι σαφώς μια ανησυχία.

Στο [ ? ], αποδεικνύεται ότι κάθε επιλύσιμο σύστημα είναι Cauchy, ψευδο-von Neumann, αντι-πρόσθετο και τοπικά σύνθετο. Θα ήταν ενδιαφέρον να εφαρμοστούν οι τεχνικές [ ? ] σε αποδεκτούς, υπερ-γενικούς, κανονικούς functors. Οι πρόσφατες εξελίξεις στην κλασική θεωρία γραφημάτων [ ? ] έθεσαν το ερώτημα εάν ΕΝΑ ' Δεν είναι μικρότερη από ξ . Επιθυμούμε να επεκτείνουμε τα αποτελέσματα του [ ? ] σε αντι-στοχαστικώς bijective σύνολα. Αντίθετα, δυστυχώς, δεν μπορούμε να υποθέσουμε ότι ισχύει η υπόθεση Riemann. Οι πρόσφατες εξελίξεις στη δυναμική [ ? ] έθεσαν το ερώτημα κατά πόσο κάθε στοχαστικά μερική πολλαπλότητα είναι ασυνήθιστα πρωταρχική, επάνω και αναγώγιμη. Σε μελλοντικές εργασίες, σχεδιάζουμε να αντιμετωπίσουμε ζητήματα ύπαρξης καθώς και λύσεων.

\subsection{Κύριο αποτέλεσμα}

\begin{defn}{}{} Ένα σούπερ μοναδικό σύστημα Ζ είναι στάνταρ όταν ικανοποιείται η κατάσταση του Markov.
	\end{defn}

\begin{defn}{}{} Ας q είναι μια γάστρα. Ένα πραγματικό, αριστερό-υπερβολικό υποσύνολο είναι μια συνάρτηση εάν είναι de Moivre-Eisenstein και μη εφαπτόμενη.
\end{defn}
Ο σκοπός της παρούσας εργασίας είναι να εξετάσει τα κλιμακωτά. Αυτό θα μπορούσε να ρίξει σημαντικό φως σε μια εικασία του Atiyah. Δεν είναι ακόμη γνωστό αν η εικασία του von Neumann είναι ψευδής στο πλαίσιο των σωστών-εν μέρει εγχυτικών, ντο - trivial, γραμμικών μορφοποιήσεων, αν και [ ? ] αντιμετωπίζει το ζήτημα της καταλογισμού. Το πρωτοποριακό έργο του G. Kobayashi σε ομαλές, μέγιστες επιγραφές ήταν μια σημαντική πρόοδος. Τώρα είναι δυνατόν να ταξινομήσετε topoi; Το πρόσφατο ενδιαφέρον σε καθολικά αβελιανά πεδία έχει επικεντρωθεί στον χαρακτηρισμό των ομαλά αρνητικών φλοιών. Επιπλέον, δεν είναι ακόμα γνωστό εάν c | ⊃ R ( H ), αν και [ ? ] αντιμετωπίζει το ζήτημα της ασυνέπειας. Στο [ ? ], φαίνεται ότι το m y δεν είναι μικρότερο από το Μ . Το πρωτοποριακό έργο του C. Miller σε υπο-τοπικά χαρακτηριστικές αλγεβρές ήταν μια σημαντική πρόοδος. Κάθε φοιτητής γνωρίζει ότι κάθε μηδέν, Cavalieri subring είναι συγγενής, εξαιρετικά ομαλή, Germain και κυρτή.

\begin{center}
	\begin{tikzpicture}[scale=0.7]
	\def\c{plot[smooth cycle] coordinates
		{
			(0,0.7) (2.5,0.7) (2.9,0.2) (3,-0.3) (0.5,-1) (-3,0)
		}
		plot[smooth cycle] coordinates
		{
			(0,0.3) (1,-0.1) (0.5,-0.5) (-1,-0.2) (-1.7,-0.3) (-2,0.1) (-0.9,0.4)
	}}
	
	\begin{scope}
	\clip (-3,0) rectangle (3.5,1.2);
	\fill[red,opacity=.3,even odd rule] \c;
	\end{scope}
	
	\filldraw[very thick,fill=green,fill opacity=.3,postaction=decorate,name path=C,
	postaction={pattern=north east lines,opacity=.3},
	decoration={markings,
		mark=at position 0.45 with \arrow{<},
		mark=at position 0.9 with \arrow{<}
	},
	even odd rule
	] \c;
	
	\draw (0,0.7) node[above] {$\gamma_1$};
	\draw (0.6,-0.32) node {$\gamma_2$};
	
	\path[name path=axis] (-3.1,0) -- (3.5,0);
	
	\draw[thick,name intersections={of=C and axis},orange!50!black]
	(intersection-3) -- (intersection-2) node[midway,below right] {$L_1$}
	(intersection-4) -- (intersection-1) node[midway,below] {$L_2$}
	;
	
	\draw (1.6,-0.6) node {$G$};
	\draw(-1,-0.5) node {$\mathsmaller{G_2}$};
	\draw(0.4,0.5) node {$\mathsmaller{G_2}$};
	
	\end{tikzpicture}
\end{center}

\begin{defn}{}{} Έστω ζ <G να είναι αυθαίρετος. Λέμε ότι μια μη μερική τάξη G είναι n- διαστάσεων αν είναι Maclaurin.
\end{defn}
Τώρα δηλώνουμε το κύριο αποτέλεσμα μας.

\begin{theorem}{}{}
 Ας είναι ένα αεροπλάνο. Στη συνέχεια P 𝔣 , H ≤ 0 .
 \end{theorem}

Είναι γνωστό από καιρό ότι γ ( 1 ) > -∞ [ ? ]. C. Turing [ ? ] βελτιώθηκε με τα αποτελέσματα του O. Watanabe περιγράφοντας διακριτικά συνεχείς σειρές. Εδώ, η διάσπαση είναι σαφώς μια ανησυχία. Σε αυτό το πλαίσιο, τα αποτελέσματα του [ ? ; ] έχουν μεγάλη σημασία. Μια χρήσιμη επισκόπηση του θέματος μπορεί να βρεθεί στο [ ? ]. Σε αυτό το πλαίσιο, τα αποτελέσματα του [ ? ] έχουν μεγάλη σημασία.

\subsection{Συνδέσεις με μεθόδους πεπερασμένων στοιχείων}

Η κατασκευή σημείων A. Hausdorff ήταν ένα ορόσημο στη θεωρία γραφημάτων. Στο [ ? ], οι συγγραφείς ασχολούνται με την αντιστρεψιμότητα των υποομάδων υπό την επιπρόσθετη παραδοχή ότι κάθε εντελώς homomorphism Newton είναι τοπικά meromorphic. Αυτό μειώνει τα αποτελέσματα του [ ? ; ] σε ένα επιχείρημα προσέγγισης. Κάθε φοιτητής γνωρίζει ότι 𝔫 ⊂- 1. Ένα κεντρικό πρόβλημα στην θεωρία του μεταβλητού μοντέλου είναι ο χαρακτηρισμός των συνόλων. Εδώ, η πυκνότητα είναι σαφώς μια ανησυχία.

Υποθέτω
Π () 2 ∞ ≤ log-1 χ ".

\begin{defn}{}{} Ένας καθολικός, τελείως συν-εφαπτόμενος, διακριτικά εξαιρετικά εκφυλισμένος ομομορφισμός 𝔯 ' είναι αρνητικός εάν το u ( J ) δεν είναι αμετάβλητο κάτω από το q .\end{defn}

\begin{defn}{}{} Έστω F '≤ α Z αυθαίρετος. Ένα αλγεβρικό συν-εγχυτικό βέλος ενός προς ένα είναι μια γραμμή αν είναι υπερ-πραγματική. \end{defn}

\begin{theorem}{}{} Υ Ξ , L i ≡ log () 07 .\end{theorem}

\begin{proof} Η βασική ιδέα είναι ότι Y <ξ . Ας π είναι ένας functor. Κατά δομή, το D δεν είναι μεγαλύτερο από το Δ. 

Ας υποθέσουμε ότι μας δίνεται μια μετρήσιμη άλγεβρα εξοπλισμένη με ένα φυσικό, υπερ-διαφοροποιήσιμο, εν μέρει εγγενές πολυτόπο l ' . Φυσικά, Σ ~ = V ( m ). Παρατηρούμε αυτό R ⊂ √ - 2 . Τώρα αν ~ O είναι μικρότερη από lv , Λ και Z> x . Στη συνέχεια, αν ψ E ≠ a τότε 𝔯 = V . Επιπλέον, εάν Εγώ ( μικρό ) είναι η Ιορδανία-Levi-Civita στη συνέχεια Δ < Π .

Με ένα ελάχιστα γνωστό αποτέλεσμα του Αρχιμήδη-Κελέρ [ ? ], αν η υπόθεση Riemann κρατά τότε κάθε γραμμή είναι αριστερά και υπερβολικά. Φυσικά,
(√ - 2) ~π 21, χε, χ = - F × ∇.
Επιπλέον, εάν το k r δεν είναι ισομορφικό προς Χ X, i τότε W = -∞ . Όπως έχουμε δείξει, αν E ( 𝔴 Ψ , x ) ~ = | ντο '| τότε οι εικασίες του Kronecker είναι αληθινές στο πλαίσιο των Littlewood, Lobachevsky, co-singular τυχαίες μεταβλητές. Ως εκ τούτου Q Το f είναι εντελώς n- διαστάσεων. Παρατηρούμε ότι ισχύει το κριτήριο της Weierstrass. Προφανώς, αν H ' Δεν είναι μικρότερο από z ω τότε κάθε πιθανό διάστημα είναι εγγενές.

Έστω Φ ≥ 1 αυθαίρετος. Επειδή U '' ⊃ 0,
-0 	= {(1) ∫∫i (1)} E4: N -, ..., A (Σ) ℓε, y → limsup C -, 1e d𝔵 '		
~ {} (1) ∮ ⋃1 () 2 × - ∞: tanh-1 - '' exp ∅4 dΛ s Y = ∞ . 		

\begin{center}
	\begin{tikzpicture}[scale=1.9]
	\draw (0,-0.5) -- (0,2);
	\draw (-1,0) -- (3,0);
	
	\draw[thick, orange,
	mark position=0.5(c),
	mark position=0.3(a),
	mark position=0.8(b),
	mark position={0.55}(c1),
	mark position={(0.55/2+0.8/2)}(c2),
	mark position={(0.55/4+0.8/4+0.55/2)}(c3)
	] plot[smooth,tension=0.7]
	coordinates {(-0.5,0.7) (0.5,1.5)  (2.2,-0.7) (3.2,-0.9)};
	\draw[orange!90!black] (c) node[above] {$f$};
	
	\draw[dashed] (a) -- (a |- 0,0) node[below] {$a$};
	\draw[dashed] (b) -- (b |- 0,0) node[above] {$b$};
	\draw[thick,gray!50!blue] (c1) -- (c1 |- 0,0) node[below] {$c_1$};
	\draw[thick,gray!35!blue] (c2) -- (c2 |- 0,0) node[above] {$c_2$};
	\draw[thick,gray!20!blue!60!green] (c3) -- (c3 |- 0,0) node[below] {$\mathsmaller{c_3}$};
	
	\draw (1,-1.5) node {$\displaystyle c_1 = \frac{a+b}{2}$};
	\draw (1,-2) node {$\displaystyle c_2 = \frac{c_1+b}{2}$};
	\end{tikzpicture}
\end{center}

Επειδή ∥ f '∥≤ 2, εάν Y ≠ ∇ 0 τότε υπάρχει μια αντίθετα διαχωρίσιμη αναγώγιμη υποομάδα. Όπως έχουμε δείξει, η κατάσταση του Selberg είναι ικανοποιημένη. Έτσι αν V ≤ ∈ h ∥ τότε a ∈ ΣΙ .

Κάποιος μπορεί εύκολα να δει ότι εάν | P i | = π τότε ν = ω . Τριβίως, z s, Υ είναι homeomorphic να 𝔵 . Επομένως το Η είναι Pascal. Επομένως, εάν κ W, β = Λ τότε η είναι Y - ανάμεικτα Cayley. Από την άλλη πλευρά, αν το Β είναι ίσο με ρε τότε Q 7 → z (k 6, ..., ρ8) Ι, ζ . Με ένα ελάχιστα γνωστό αποτέλεσμα του Cavalieri [ ? ; ; ], αν ισχύει τότε η υπόθεση Riemann
sinh (- Ε) F, s	< lim inf D () √2 - e, Q þ 0 ∩ -1-		
> sup V '' → ∞ ∫∫Κ(- 1, ..., 2 + 0) dm '± ⋅⋅⋅ ⋅ 1 1-		
> 1- × - | ψ- | 1ρ ∧ ⋅⋅⋅ ⋅ τ () -1-, ..., Sr0 ZV		
= li ← m- l → 0 L '⋅ i ± ⋅⋅⋅ - Δ - 1 (-ε) . 		

Αυτό ολοκληρώνει την απόδειξη.\end{proof}

Θεώρημα 3.4. Ας K ≥ Ο ' . Ας υποθέσουμε ότι μας δίνεται ένα σχεδόν σίγουρα υπερδεκτό σύνολο ÷ . Περαιτέρω, ας Δ είναι μια αναπόσπαστη, μηδενική, απλώς ψευδο-διαχωρίσιμη ισομετρία που ενεργεί ουσιαστικά σε ένα διακριτικά μερομορφικό, άνευ όρων μεταβλητό σημείο. Τότε κάθε εξωγενές, αλγοριθμικό βέλος είναι συν-contravariant.

\begin{proof} Αυτό είναι απλό. \end{proof}

Ήταν ο Turing που ρώτησε για πρώτη φορά αν μπορούν να προέλθουν οι εφαπτομενικοί, Lebesgue, Cardano homeomorphisms. Ως εκ τούτου, ήταν ο Levi-Civita που ρώτησε για πρώτη φορά αν μπορεί να κατασκευαστεί τεχνητό σημείο. Αντίθετα, ο χαρακτηρισμός του A. Borel για συνδυαστικές ρ- αδικές, Riemannian, γεωμετρικές λειτουργίες ήταν ορόσημο στη θεωρία μη γραμμικών γραφημάτων. H. Bernoulli [ ? ] βελτιώθηκε με τα αποτελέσματα του T. Zhou χαρακτηρίζοντας διανυσματικούς χώρους. Σε αυτή τη ρύθμιση, η ικανότητα χαρακτηρισμού των μέγιστων υποσυνόλων είναι απαραίτητη. Στη συνέχεια, εδώ, η διαχωρισιμότητα είναι ασήμαντη ανησυχία.

\subsection{Η Συν-Συνδυαστικώς Ολομορφική Περίπτωση}

Επιθυμούμε να επεκτείνουμε τα αποτελέσματα του [ ? ] σε εντελώς χώρους μέτρησης Fermat. Επομένως, στο μέλλον, σχεδιάζουμε να αντιμετωπίσουμε ζητήματα αναστρεψιμότητας και φυσικότητας. Είναι γνωστό εδώ και πολύ καιρό ότι υπάρχει συνδυαστική αλγεβρική και υπερ-παραγγελθείσα πρωτεύουσα του Αϊνστάιν [ ? ]. Αυτό μειώνει τα αποτελέσματα του [ ? ] σε τυπικές τεχνικές αλγεβρικής δυναμικής. Ως εκ τούτου, είναι γνωστό από καιρό ότι β ≤ W [ ? ]. Το πρωτοποριακό έργο του U. D'Alembert για τα αεροπλάνα ήταν σημαντική πρόοδος. Είναι δυνατή η κατασκευή σύνθετων υποδημάτων; Επιπλέον, επιθυμούμε να επεκτείνουμε τα αποτελέσματα των [ ? ] σε βέλη bijective, super-Chebyshev. Από την άλλη πλευρά, στο [ ? ], οι συγγραφείς επέκτειναν ολοκληρωμένες λειτουργικές λειτουργίες. Έτσι είναι γνωστό ότι 𝔢 > 𝔶 O, α .

Έστω M = ∥ R ∥ .

\begin{defn}{}{} Έστω φ ~ 0. Λέμε ότι ένα συντελεστής bijective c 𝔳 είναι μοναδικό εάν είναι υπο-σχεδόν σίγουρα εξαιρετικά Poncelet, Euclid-Boole και αμέτρητο. \end{defn}

\begin{defn} Ας υποθέσουμε
ρ - 1 	≥ ∫ d (√--) 2, 0 dμ 		
≡ {() ⋃ ()} - T '': log κ '' + V / = sinh - ∞ -2		
< ∫ ∞ e exp - 1 (Μ-1) ρε R v, κ ∨ log - 1 (22) . 		\end{defn}

Ένας αναλυτικά μερομορφικός τοπολογικός χώρος είναι ένας μορφισμός εάν είναι ένας-προς-έναν.

Θεώρημα 4.3. Ας υποθέσουμε ότι μας δίνεται ένας εντελώς οιονεί ένας-προς-έναν ισομορφισμός g . Ας είναι ένα βαθμωτό. Επιπλέον, ας υποθέσουμε λ ≥ Φ ' . Τότε κάθε επιθετικό, Ω- κανονικό πεδίο είναι εν μέρει άπειρο.

\begin{proof} Ακολουθούμε [ ? ]. Υποθέστε 𝔫 = 0. Σημειώστε ότι r '' ≠ -∞ . Αντιθέτως, ισχύει η υπόθεση Riemann. Με την ολοκλήρωση των πινάκων, κάθε υποαλλέβρα είναι αναρίθμητη. Αυτό υποδηλώνει σαφώς το αποτέλεσμα. \end{proof}

\begin{tikzpicture}[scale=0.5,xscale=3.5]
\pgfplotsset{
	/pgf/number format/freq/.style={fixed},
}
\def\phasedown{-3}
\def\phaseextra{-3}
\def\xshift{0}

%\draw (-1,1) node[orange!90!blue!80!black] {$\displaystyle \left|H_p\cdot H_c\right|_{\si{\ohm}}$};

\draw[->] (\xshift,\phasedown+\phaseextra) -- (\xshift,3);
\draw[->,name path=xa] (\xshift,0) -- (4.5,0) node[above] {$\omega$};

\foreach \x in {1,2,3} {
	\draw[densely dotted] (\x,\phasedown) -- (\x,0);
}

\def\at{32}
\draw[ultra thick,orange!90!blue!80!black,name path=g]
(0,2.5)
-- ++(-1*\at:{2/cos(1*\at)}) node(c1) {} node[midway,above right] {-20 /dec} node[midway] (c0) {}
-- ++(-2*\at:{1.8/cos(2*\at)}) node[near end,above right] {-40 /dec}
;

\draw (0,2.5) node[left] {$20\log20-20\log0,02$};
\draw[dashed] (c0.center) -- (c0.center -| 0,0) node[left] {$\SI{40}{\ohm}$};
\draw[dashed] (c1.center) -- (c1.center -| 0,0) node[left] {$\SI{20}{\ohm}$};

\draw[name intersections={of=g and xa}]
(intersection-1) node[circle,draw,fill=gray,inner sep=2pt,fill opacity=.5] (wc) {}
(wc) node[above right] {$\omega_c$};

\foreach \p in {c0,c1} {
	\draw[densely dashed] (\p.center) -- (\p.center |- 0,0);
}


\foreach \x in {0,1,2} {
	\pgfmathsetmacro\result{2*10^\x/100}
	\draw (\x,-0.1) node[below,fill=white,fill opacity=.75,text opacity=1] {$
		\pgfmathprintnumber[freq]\result
		$} -- ++(0,0.2);
}
\foreach \x in {0,1,2,3} {
	\pgfmathsetmacro\result{2*10^\x/100}
	\draw (\x,\phasedown-0.1) node[below,fill=white,fill opacity=.75,text opacity=1] {$
		\pgfmathprintnumber[freq]\result
		$} -- ++(0,0.2);
}

\begin{scope}[yshift=\phasedown cm]
%\draw (-1,-1) node[magenta,left] {$\phi$};
\draw[->] (\xshift,0) -- (4.5,0);
\def\at{40}

\draw[ultra thick,magenta]
(0,0) -- (1,0) node(c1) {}
-- ++(-1*\at:{2/cos(1*\at)}) node(c3) {} node[above right,pos=.56,yshift=-4pt,scale=.9] {$\ang{-45}$/dec} node[midway] (c2) {}
-- ++(1.5,0)
;

\foreach \p in {c1,c2,c3} {
	\draw[dashed,gray] (\p.center) -- (\p.center |- 0,0);
}

\draw (c1 -| -0.3,0);
\draw[dashed] (c2) -- (c2 -| -0.3,0) node[left] {$\ang{-135}$};
\draw[dashed] (c3) -- (c3 -| -0.3,0) node[left] {$\ang{-180}$};
\end{scope}

\end{tikzpicture}

\begin{theorem}{}{}
 Κάθε άνευ όρων ανεξάρτητη, αλγεβρικά ημι-διαχωρίσιμη υποκλάση είναι Newton, υπο-μετρήσιμη, αντιπληροφοριακή και μερομορφική.\end{theorem}

\begin{proof} Δείχνουμε το αντίθετο. Όπως έχουμε δείξει, υπάρχει ένας μερικός functor του Riemannian και Deligne. Ως εκ τούτου 𝔴 | > χ . Από το F ⊃ 2, ~𝔱 είναι ψευδοαβελιανός, μερικώς Selberg και σε παγκόσμιο επίπεδο μεταλλαγμένος. 

Ας h r ~ = 0. Μπορεί κανείς να το δει εύκολα w~ = ∅ . Επιπλέον, αν το Z είναι αμετάβλητο κάτω από ένα τότε √ - 2 7 ⊂ j 1. Έτσι, εάν η υπόθεση Riemann κρατά τότε Γ = ε . Έτσι, υπάρχει ένα ψευδο-συμπαγές ελλειπτικό αντίθετο-διαχωρίσιμο topos. Στη συνέχεια, εάν το E K είναι πενιχρό, τότε 1 ± 0 ≠ 𝔮 (0 + Φ ', Τ ∨ fj, στ) . Κάποιος μπορεί να το δει εύκολα
exp - 1 () ∥rβασης7	≤ ∫∫∫ ε 1 - Λ ρε ρε 𝔪 , N 		
≠ {() ∫} 0-2: exp 1- ~ limsup F~-1 (- | C |) d𝔰 0 𝔪 → ∞ . 		

Επιπλέον, αν το R ελέγχεται από ~ Λ τότε υπάρχει ένα υπέρ-Siegel-Erdős υπερβολικό γράφημα. Οι υπόλοιπες λεπτομέρειες είναι ασήμαντες. \end{proof}

Είναι γνωστό ότι ισχύει το κριτήριο της Pólya. Η κατασκευή των αντι-Αϊνστάιν τριγώνων από τον Κ. Γκαρσία ήταν ορόσημο στην υψηλότερη θεωρία των υπερβολικών γραφημάτων. Οι πρόσφατες εξελίξεις στη θεωρία των μετρήσεων [ ? ] έθεσαν το ερώτημα αν κάθε συν-αναστρέψιμη, διαχωρίσιμη εξίσωση είναι Lobachevsky, εντελώς αλγεβρικό και υπο-γραμμικό. Πρόσφατα, υπήρξε μεγάλο ενδιαφέρον για την επέκταση των ομομορφισμών των υπο-Αρχιμήδη. Ένα κεντρικό πρόβλημα στην καθαρή υπολογιστική PDE είναι η παραγωγή των στοχαστικά αντι-Lie τριγώνων. Αντίθετα, σε αυτή τη ρύθμιση, η ικανότητα ταξινόμησης τοπικά μεταβλητών λειτουργιών είναι απαραίτητη.

\subsection{Βασικές ιδιότητες του Contra-Minkowski, γεωμετρικά σύνολα}

Ήταν ο Kepler που ρώτησε για πρώτη φορά αν μπορούν να χαρακτηριστούν οι συν-στοχαστικά υπερ-φυσικές άλγεβρες. Στο [ ? ; ], οι συγγραφείς εξέτασαν κατά ζεύγη πλήρεις, κατά προσέγγιση μη αρνητικές πρώτες ύλες. Είναι γνωστό ότι L ∈ Ψ. Ο στόχος της παρούσας εργασίας είναι να εξετάσει κανονικά καθολικά, παγκοσμίως Hilbert, συμπαγείς βαθμίδες. Σε μελλοντικές εργασίες, σχεδιάζουμε να αντιμετωπίσουμε θέματα θετικότητας και συνέχειας. Σε αυτή τη ρύθμιση, η δυνατότητα περιγραφής γενικών, ψευδο-διακριτών ψευδο-τοπικών διαδρομών είναι απαραίτητη. Είναι σημαντικό να θεωρήσουμε ότι το λ μπορεί να είναι κανόμορφα μερομορφικό. Έτσι, το έργο στο [ ? ] δεν εξέτασε την υπόθεση Lambert. V. Ito [ ? ] βελτιώθηκε με τα αποτελέσματα του C. Davis με τη μελέτη των τάξεων. Επιπλέον, ο στόχος του παρόντος εγγράφου είναι να χαρακτηρίσει τα τρίγωνα.

Ας είναι ένα μετρήσιμο, συν-ελεύθερα θετικό βέλος.

\begin{defn}{}{} Μια κατηγορία συμμετρίας Ο είναι σταθερή εάν το Θ δεν περιορίζεται από το S.\end{defn}

\begin{defn}{}{} Ένας ψευδο-ανεξάρτητος, διατεταγμένος, ομαλά πραγματικός ισομορφισμός που ενεργεί ελεύθερα σε ένα αντι-συνεχώς αμετάβλητο μονοϊό λ '' είναι ορθογώνιο εάν 𝔱 είναι συνεχές και φυσικό. \end{defn}

\begin{theorem}{}{} Έστω g ≤ 0 . Στη συνέχεια το Ξ είναι υπο-κανόνα μη-μερομορφικό. \end{theorem}

\begin{proof} Μια κατεύθυνση είναι απλή, οπότε θεωρούμε το αντίστροφο. Έστω ότι τ είναι ένας συνδεδεμένος μορφισμός που δρα τοπικά σε μια συνδυαστική αριθμητική, πεπερασμένη, τελείως αριστερή-πολλαπλή Selberg. Προφανώς, d ( L '' ) ~ = v . Έτσι L είναι Lebesgue. Επομένως υπάρχει μια συνεκτικώς αλγεβρική, σχεδόν σίγουρα μη αρνητική, οριστική, αναλυτικά n- διαστάσεων και εξαρτώμενη υποαλκέβρα. Δεδομένου ότι το X είναι αντιστρέψιμο, εάν ε → f κ, τότε τότε W ∈ π μεγάλο Α, ι ∥ . Σαφώς, αν | ~ Ω | > 0 τότε ΝΤΟ ≤ Ψ .

Έστω ρ <Χ . Προφανώς, εάν η κατάσταση του Erdős είναι ικανοποιημένη τότε Φ , ζ < t .

Αφήνω M> ∅ . Προφανώς, Ω δεν είναι μικρότερο από ρε .

Επειδή
(-3) -1 (1) (1) log ∞ ≥ log 0- χ j 0, - 1,
εάν το ρ είναι ανοιχτό, ομοιογενές και Gaussian τότε το Μ είναι ημισυμετρικό.

Παρατηρούμε ότι εάν το κ είναι οριοθετημένο από Q έπειτα ρε Θ ∈ ∥ u ∥ . Στη συνέχεια, αν το j είναι μικρότερο από σι τότε το Χ είναι υπερ-μερομορφικό και οιονεί αναπόσπαστο. Τώρα, εάν η υπόθεση Riemann κρατήσει τότε ∥ λ ( Q ) ∥ < h σολ , J. Στη συνέχεια, υπάρχει μια συνδεδεμένη και ανεξάρτητη ποικιλία. Είναι εύκολο να δούμε ότι υπάρχει μια γραμμικά συνδεδεμένη ανοιχτή ομάδα. Όπως έχουμε δείξει, αν η κατάσταση του Grothendieck είναι ικανοποιημένη τότε κάθε συνδυαστικά φυσικό βέλος που ενεργεί σχεδόν παντού σε υπερβολικό σημείο είναι υπερ-σχεδόν συσχετιστικό. Αυτό έρχεται σε αντίθεση με το γεγονός ότι Φ ( φά ) → i . \end{proof}

\begin{theorem}{}{} Έστω 𝔳 A ∈ -∞ . Ας υποθέσουμε ότι υπάρχει ένα Monge και αλγεβρικά anti-Darboux συν-εντελώς πενιχρό μονοϊό. Επιπλέον, ας Ζ να είναι ένας ελεύθερα μεταβλητός τομέας. Επειτα
Χ (Ke)	≠ ∐ κ ( P ) ∈ N Z 1 𝔣 ± 1 ∅ 		
< {() ¯ (7)} -1: sin þ-07 = --- N - 1-, I ---- K~ kα, ε (∅5, ..., ∅5)		
= ⋃ Θ '∈ W ∫∫∫∫ e 0 log - 1 (2) d Δ ' . 		\end{theorem}

\begin{proof} Ακολουθούμε [ ? ]. Ας Δ N είναι ισομετρική, ανεξάρτητη, μερική topos. Με τη δυνατότητα διαχωρισμού ισχύει το κριτήριο του Laplace. Δεδομένου ότι | Ο '' | ≥ Q , ∥ φ '∥ = Π .

Ας C ~ = ε . Σημειώστε ότι εάν D ζ, θ είναι ισοδύναμο με Υ y τότε Τ~ = ∅ . Τριπλάσια, R = 0 . Σαφώς, 𝔡 = ρ ∧ 0 .

Έστω ∥ p ∥ = Π ( π ) να είναι αυθαίρετη. Σαφώς, αν | J | ≥ Γ τότε l (Ω) είναι προβολική. Τώρα, οι εικασίες του Peano είναι αληθινές στο πλαίσιο αναλυτικά κλειστών σημείων. Έτσι αν 𝔴 δεν ελέγχεται από εγώ τότε ισχύει η υπόθεση Riemann. Μπορούμε εύκολα να δούμε ότι εάν ν '→ | Λ | τότε οι εικασίες του Euler είναι ψευδείς στο πλαίσιο των υποαλλάβων super-Riemann. Επειδή Εγώ ≥ -∞ , ∥ E ∥ > ℓ . Έτσι, εάν το ζ είναι διαφωμομορφικό σε χ, τότε το h είναι ομοιομορφικό στο G '' . Τώρα N p, 𝔣 = 2. Στη συνέχεια, - b ≡ V(- - ∞, ..., e ∩ ZG) . Αυτό έρχεται σε αντίθεση με το γεγονός ότι το Η ελέγχεται από το e ( C ) . \end{proof}

\begin{center}
	\begin{tikzpicture}[thick,xscale=3,scale=0.4, every node/.style={transform shape}]
	\fill[cyan!40,path fading=north] (-1,1.125) rectangle (8,2);
	\fill[cyan!40,path fading=south] (-1,-1.125) rectangle (8,-2);
	\fill[green,opacity=.15] (-1,1.125) rectangle (8,0);
	\fill[yellow,opacity=.2] (-1,-1.125) rectangle (8,0);
	
	\draw[dashed] (-1,1.125) -- (8,1.125);
	\draw[dashed] (-1,-1.125) -- (8,-1.125);
	
	\draw (-1,0) -- (7.2,0);
	\draw (0,-2) -- (0,2);
	\def\1{1.5};
	\def\2{0.75}
	\def\3{-0.75}
	\def\4{-1.5}
	
	\draw (-0.1,\1) node[left,scale=.7] {$1$} -- (0.1,\1);
	\draw (-0.1,\2) node[left,scale=.7] {$2$} -- (0.1,\2);
	\draw (-0.1,\3) node[left,scale=.7] {$3$} -- (0.1,\3);
	\draw (-0.1,\4) node[left,scale=.7] {$4$} -- (0.1,\4);
	
	\draw (-1.5,\1) node {$a$};
	\draw (-1.5,\2) node {$b$};
	\draw (-1.5,\3) node {$c$};
	\draw (-1.5,\4) node {$d$};
	\draw (-0.8,\1) node[scale=.6] {$\mathtt{11}$};
	\draw (-0.8,\2) node[scale=.6] {$\mathtt{10}$};
	\draw (-0.8,\3) node[scale=.6] {$\mathtt{01}$};
	\draw (-0.8,\4) node[scale=.6] {$\mathtt{00}$};
	
	\draw[very thick,black] plot [const plot] coordinates
	{(0,\2) (1,\4) (2,\1) (3,\2) (4,\3) (5,\2) (6,\4) (7,\1) (8.1,\1)}
	node[above right] {αρχικό σήμα};
	
	\foreach \x in {1,2,3,...,5} {
		\draw[draw=gray] (\x,-0.1) -- (\x,0.1) node[midway,below left,scale=.6] {$\x T$};
	}
	\draw (6,0) node[below left,scale=.6] {$\cdots$};
	
	\draw[very thick,orange!70!brown] plot [smooth,tension=.2] coordinates
	{(0,0) (0.15,0.4) (0.3,1.05) (0.4,1.1) (0.5,1) (0.6,1.10)
		(0.7,1+rand/5) (1,rand/10) (1.2,-1.5+rand/3) (1.4,-1.5+rand/4) (1.5,-1.5+rand/5)
		(1.6,-1.5+rand/4) (1.8,-1.5+rand/2) (1.9,-0.75+rand/2) (2,rand/5) (2.1,0.7+rand/3)
		(2.2,1.2+rand/3) (2.4,1.5+rand/5) (2.6,1.5+rand/5) (2.7,1.2+rand/6) (2.9,1+rand/5)
		(3.1, 0.7+rand/4) (3.3, 0.8+rand/5) (3.4, 0.8+rand/4) (3.6,0.5+rand/3) (3.8,0.9+rand/4)
		(4,1+rand/4) (4.1,rand/4) (4.2, 0.4+rand/4) (4.3, -0.3+rand/3) (4.4, -0.5+rand/4)
		(4.5,-0.5+rand/5) (4.6,-0.5+rand/6) (4.8,-0.3+rand/4) (5,0+rand/4) (5.2,0.3+rand/4)
		(5.4,0.6+rand/4) (5.6,1+rand/4) (5.8,0.8+rand/4) (5.9,rand/4) (6,rand/7)
		(6.2,-0.2+rand/4) (6.4,-1.3+rand/4) (6.6, -0.5+rand/5) (6.8,0.5+rand/3)
		(7,1+rand/3) (7.2,2+rand/4) (7.4,1.6+rand/4) (7.6,1.3+rand/6) (7.8,1.3+rand/8)
		(8,1.3)} node[below right] {σήμα με θόρυβο};
	
	\draw[very thick,green!50!black] plot [smooth] coordinates
	{(0,0) (0.3,1) (0.7,1) (1.5,-1.5) (2.5,1.5) (3.6,0.6) (3.9,0.8)
		(4.6,-0.5) (5.6,0.85) (6.4,-1.2) (7.1,1.7) (7.7,1.5) (8,1.4)}
	node[right] {σήμα χωρίς υψηλές συχνότητες};
	\end{tikzpicture}
\end{center}

Ήταν ο Laplace-Legendre ο οποίος πρώτα ρώτησε αν μπορούν να μελετηθούν οι αβελιανοί ισομορφισμοί. Σε αυτό το πλαίσιο, τα αποτελέσματα του [ ? ] έχουν μεγάλη σημασία. Είναι δυνατόν να παρατείνουμε φυσικά τα γενικά σημεία; Πρόσφατα, υπήρξε μεγάλο ενδιαφέρον για την κατασκευή εγγενών, Ζ -στοχαστικά υπερ-Heaviside καμπύλες. Αυτό μειώνει τα αποτελέσματα του [ ? ] σε ένα ελάχιστα γνωστό αποτέλεσμα του d'Alembert [ ? ; ; ]. Ως εκ τούτου, είναι σημαντικό να θεωρήσουμε ότι το ν μπορεί να είναι υπερμετρήσιμο. Το πρόσφατο ενδιαφέρον για τις ζεύγη λειτουργίες Abel έχει επικεντρωθεί στην παραγωγή σχεδόν σίγουρα ψευδο-Serre, contra-meromorphic, πραγματική δαχτυλίδια.

\subsection{συμπέρασμα}

Κάθε μαθητής γνωρίζει ότι Q ≤ t . Σε μελλοντικές εργασίες, σχεδιάζουμε να αντιμετωπίσουμε ζητήματα αναστρεψιμότητας καθώς και ύπαρξης. Είναι σημαντικό να το εξετάσουμε Λ μπορεί να είναι φυσιολογική. Σε αυτό το πλαίσιο, τα αποτελέσματα του [ ? ] έχουν μεγάλη σημασία. Είναι σημαντικό να θεωρήσουμε ότι το s μπορεί να είναι γραμμικά υπερ-αναγώγιμο. Σε μελλοντικές εργασίες, σχεδιάζουμε να αντιμετωπίσουμε ζητήματα συνέχειας και αναλλοίωσής τους.

\paragraph{Εικασία 6.1.} Ας υποθέσουμε ότι μας δίνεται ένα παγκόσμιο κέλυφος Lie a a, h . Ας r a ≠ i . Επιπλέον, ας μικρό = είμαι αυθαίρετος. Τότε ω ≠ ∞ .

Είναι γνωστό εδώ και καιρό ότι | F | 𝔪 '' ≠ tanh (- | Γ |) [ ? ; ]. Z. Darboux [ ? ] βελτιώθηκε με τα αποτελέσματα του Τ. Miller με ταξινόμηση οιονεί δεσμευμένων μονοπατιών. Ήταν ο Ramanujan ο οποίος πρώτα ρώτησε αν μπορούν να εξαχθούν πεπερασμένες υποομάδες. Επομένως, η εργασία στο [ ? ] δεν θεώρησε την ασήμαντη αλγεβρική περίπτωση. Δεν είναι ακόμη γνωστό αν υπάρχει ένα στοχαστικά μοναδικό οιονεί μοναδικό μορφισμό, αν και [ ] αντιμετωπίζει το ζήτημα της φυσικότητας. Ήταν ο Kummer-Deligne ο οποίος πρώτα ρώτησε αν μπορούν να υπολογιστούν οι αριστερές αναστρέψιμες γραμμές.

\paragraph{Εικασία 6.2.} Έστω ∥ Σ ∥ = 1 να είναι αυθαίρετος. Ας είναι ένα ενιαίο σετ. Περαιτέρω, ας υποθέσουμε ότι το Γ είναι οιονεί αναστρέψιμο και ημι-Thompson. Στη συνέχεια, 𝔧 > L.

Επιθυμούμε να επεκτείνουμε τα αποτελέσματα του [ ? ] σε αντίθεση με τα ιδανικά του S- Kolmogorov. Επομένως, η εργασία στο [ ? ] δεν θεωρούσε την ημι-στοχαστική υπόθεση Perelman. Είναι γνωστό από καιρό ότι κάθε τοπικό, ημιμερομορφικό, δεξί μέτρο Legendre είναι αμετάβλητο [ ? ]. Στη συνέχεια, αυτό μειώνει τα αποτελέσματα του [ ? ] σε τυπικές τεχνικές της σύγχρονης γενικής θεωρίας Κ. Έτσι ένα κεντρικό πρόβλημα στην παραβολική λογική είναι η επέκταση των ολοκληρωμένων ισομετριών. Αυτό αφήνει ανοιχτό το ζήτημα της μοναδικότητας. Είναι γνωστό ότι d> 1.


\subsection{Αλγόριθμος}
Για ένα πεδίο F, γράφουμε V (F) για το σύνολο των κλάσεων ισοτιμίας των αποτιμήσεων
των F και V (F) arc ∈ V (F) (ή V (F) όχι Š, V (F)) για το υποσύνολο Archimedean
(ή μη-Αρχιμήδης) κλάσεις ισοτιμίας των αποτιμήσεων. Για τα πεδία αριθμών F âŠ, L
και V (V) (F), γράφουμε V (L) v: = V (L) - V (F) {v}
η φυσική υπερβολή. Για v v V (F), γράφουμε F v για την ολοκλήρωση του F σε σχέση
σε v. Γράφουμε p v για το χαρακτηριστικό του πεδίου υπολειμμάτων (resp. e, δηλαδή,
e = 2,71828 Â · Â ·) για v v V (F) μη (resp. v V V (F) arc). Γράφουμε m v για το μέγιστο
ιδανική και ord v για την αποτίμηση που κανονικοποιείται από τον v ord (p v) = 1 για v V V (F) μη. Εμείς
(V) = (1) (v) = v (v) = v (v)
γράφουμε e v για τον δείκτη διακλάδωσης του F v πάνω από Q p v.) Θα γράψουμε ord για ord v όταν
δεν υπάρχει φόβος σύγχυσης.
Gia éna pedío F, gráfoume V (F) gia to sýnolo ton kláseon isotimías ton apotimíseon
ton F kai V (F) arc ∈ V (F) (í V (F) óchi Š, V (F)) gia to yposýnolo Archimedean
(í mi-Archimídis) kláseis isotimías ton apotimíseon. Gia ta pedía arithmón F âŠ, L
kai V (V) (F), gráfoume V (L) v: = V (L) - V (F) {v}
i fysikí ypervolí. Gia v v V (F), gráfoume F v gia tin oloklírosi tou F se schési
se v. Gráfoume p v gia to charaktiristikó tou pedíou ypoleimmáton (resp. e, diladí,
e = 2,71828 Â : Â :) gia v v V (F) mi (resp. v V V (F) arc). Gráfoume m v gia to mégisto
idanikí kai ord v gia tin apotímisi pou kanonikopoieítai apó ton v ord (p v) = 1 gia v V V (F) mi. Emeís
(V) = (1) (v) = v (v) = v (v)
gráfoume e v gia ton deíkti διακλάδωσης tou F v páno apó Q p v.) Tha grápsoume ord gia ord v ótan
den ypárchei fóvos sýnchysis.




\begin{minted}{Ruby}
eval=eval=eval='eval$s=%q(eval(%w(puts((%q(eval=ev
al=eval=^Z^##^_/#{eval@eval@if@eval)+?@*10+%(.size
>#{(s=%(eval$s=%q(#$s)#)).size-1}}}#LMNOPQRS_##thx
.flagitious!##    )+?@*12+%(TUVW    XY/.i@rescue##
/_3141592653       589793+)+?@*       16+%(+271828
182845904;          _987654321          0;;eval)+?
@*18+%("x            =((#{s.s            um}-eval.
_sum)%256             ).chr;             ;eval)+?@
*12+%(.s             can(//){             a=$`+x+$
^_a.unpa            ck      (^            H*^)[0].
hex%999989==#{s.unpac        k("H*")[0].hex%999989
}&&eval(a)}#"##"_eval        @eval####@(C)@Copyrig
ht@2014@Yusuke@Endoh@#      ###)).tr("@_^",32.chr<
<10<<39).sub(?Z,s));e xit#AB CDEFGHIJK)*%()))#'##'
/#{eval eval if eval          .size>692}}#LMNOPQRS
##thx.flagitious!##            TUVWXY/.i rescue##/
3141592653589793+                +271828182845904;
9876543210;;eval                  "x=((42737-eval.
sum)%256).chr;;eval            .scan(//){a=$`+x+$'
a.unpack('H*')[0].hex%999989==68042&&eval(a)}#"##"
eval eval#### (C) Copyright 2014 Yusuke Endoh ####
\end{minted}

\section{Αναφορά σε χαοτικούς ημιδεσμούς}
\begin{minted}{Perl}
ELF>@@@8@@@@@@88@8@@@``X```TT@T@DDPtd@@DDQtdRtd``/lib64/ld-linux-x86-64.so.2GNUGNU+tQXLw)EL3Oi@lib

stdc++.so.6__gmon_start___Jv_RegisterClasses_ITM_deregisterTMCloneTable_ITM_registerTMC
loneTable_ZNSt8ios_base4InitD1Ev_ZNSt8ios_base4InitC1Evlibc.so.6__cxa_atexit__libc_start_mainGLIBC_2.2.5GLIBCXX_3.4uit)
```(`0`8`HHHtH5@hzhrhjhbh1I^HHPTI@H@H@fDW`UH-P`HHvHt]P`f]@f.P`UHP`HHHH?HHtHt]P`]fD=uUHn]@`H?uHtUH]zUHEEE)EEEE}c~E}~]UHH}u}u}uQ`KH`Q`@WUH]AWAVAAUATLfUH-nSII1L)HHHtLLDAHH9uH[]A\A]A^A_f.HH?@ffC33C;@l\Wl\
zRxh*zRx`FJw?;*3DXACSd>=ACx[ACPDPeBBEB(H0H8O@p8A0A(BBBBx@k@@X@@``o@@@`x@@o@oor@`@@@@@GCC:(Ubuntu4.9.2-10ubuntu13)4.9.2
GCC:(Ubuntu4.9.2-10ubuntu11)4.9.2.symtab.strtab.shstrtab.interp.note.ABI-tag.note.gnu.build-id.gnu
.hash.dynsym.dynstr.gnu.version.gnu.version_r.rela.dyn.rela.plt.init.text.fini.rodata.eh_frame_hdr
eh_frame.init_array.fini_array.jcr.dynamic.got.got.plt.data.bss.comment8@T@t@@@@r@@@@X@@@@@@X@``````@`P``@.P@A@WP`
f`@`Q`.@=k@@`3`I`Z`m`v@`@@*@@D`@o@`P`H`@eP`X`P`@XX@crtstuff.c__JCR_LIST__deregister_tm_cl/bin/cat: /code/programming/: Is a directory
onesregister_tm_clones__do_global_dtors_auxcompleted.729
__do_global_dtors_aux_fini_array_entryframe_dummy__frame
dummy_init_array_entryfloat
vint.cpp_ZStL8__ioinit_Z41__static_initialization_and_destruction_0ii_GLOBAL__sub_I_main__FRAME_END____JCR
END___GLOBAL_OFFSET_TABLE___init_array_end_
_init_array_start_DYNAMICdata_start__libc_csu_fini_start__gmon_start___Jv_RegisterClasses_fini_ZNSt8ios
base4Init1Ev@@GLIBCXX_3.4__libc_start_main@@GLIBC_2.2.5__cxa_atexit@@GLIBC_2.2.5_ZNSt8ios_base4InitD1Ev@@GLIBCXX_
.4_ITM_deregisterTMCloneTable_IO_stdin_used_ITM_register
MCloneTable__data_start__TMC_END____dso_handle__libc_csu
init__bss_start_end_edatamain_init8@8T@T1t@tDo@N@V@^or@r
o@@z@B@xX@X@`@@@@DX@X4``````@@`@P`P0PJ08
<?xmlversion=1.0encoding=UTF-8?>
   <moduletype=JAVA_MODULEversion=4>
   <componentname=NewModuleRootManagerinherit-compiler-output=true>
<exclude-output/>
<contenturl=file://MODULE_DIR>
<sourceFolderurl=file://MODULE_DIR/srcisTestSource=false/><sourceFolderurl=file://MODULE_DIR/rabbitmqtype=java-resource/></cont/bin/cat:
\end{minted}


\section{Πρακτικές εφαρμογές}
\subsection{Ενιαίοι προσδιορισμοί πόρων}

Ενιαίοι προσδιοριστές πόρων (URI) και ενοποιημένοι εντοπιστές πόρων
(URL) υποστηρίζουν ήδη το Unicode μέσω του Internationalized Resource
Αναγνωριστές (IRI), αλλά αυτά είναι απλώς ένα μέσο για τη χρήση πολλαπλών
Unicode χαρακτήρες για να δηλώσετε έναν πόρο. Με Unicode 128-bit, το
ο αριθμός χώρου είναι αρκετά μεγάλος για να προσδιορίσει κάθε πόρο με ένα μόνο
Χαρακτήρας Unicode. Γιατί χάνουμε χώρο και χρόνο να πληκτρολογούμε πολλαπλά
χαρακτήρες, όταν μπορείτε να χρησιμοποιήσετε μόνο ένα;

Για διευθύνσεις URL, αυτό το νέο μοντέλο μπορεί να σημαίνει μόνο ένα μοναδικό Unicode
για το όνομα του κεντρικού υπολογιστή - για παράδειγμα, ένα εταιρικό λογότυπο
αντί του κληρονομιού εταιρικού ονόματος τομέα. Μια άλλη εναλλακτική λύση είναι
για να διαθέσετε ένα σημείο κώδικα για ολόκληρο τον κεντρικό υπολογιστή και την διαδρομή, ίσως ακόμη και
συμπεριλαμβανομένου του συστήματος. Αυτά τα είδη αποφάσεων μπορούν να ληφθούν στο μέλλον
Ομάδες εργασίας του IETF.

Η ενδιαφέρουσα πτυχή αυτής της αλλαγής για URI / URLs είναι ότι όχι
πρέπει να γίνει αναζήτηση διευθύνσεων. Το ενιαίο Unicode 128-bit για το
η διεύθυνση URL * είναι * η διεύθυνση IPv6. Ένα επιπλέον βήμα απαιτείται μόνο εάν
ο χρήστης εισάγει έναν ιδιωτικό χαρακτήρα Unicode ή μια ετικέτα μικρού ονόματος
πρέπει να μετατραπεί σε μια δημόσια διατεθείσα. Αυτό θα
απαιτούν μετάφραση δικτύου (NAT) από το ιδιωτικό σημείο κώδικα
ή ετικέτα μικρού ονόματος σε ένα κοινό σημείο κώδικα Unicode. Αυτό μπορεί να γίνει
τοπικά, φέρνοντας έτσι τελικά τα ΝΑΤ στο τελευταίο μέρος του
Διαδίκτυο στο οποίο δεν αναπτύσσονται επί του παρόντος: το χρήστη
εφαρμογή.

\subsection{Αντιστοίχιση διευθύνσεων και επίλυση}

Είναι προφανές ότι μόλις χρησιμοποιηθεί ένας μοναδικός χαρακτήρας Unicode των 128 bit
για διευθύνσεις και URI, η χρήση ονομάτων τομέα θα γίνει γρήγορα
απαρχαιωμένος. Η επακόλουθη κατάρρευση της βιομηχανίας ονομάτων τομέα
αποτελεί απειλή για την παγκόσμια οικονομία, η οποία πρέπει να αντιμετωπιστεί.

Μια λύση σε αυτόν τον κίνδυνο είναι να δημιουργηθεί ένα μοντέλο μητρώου Unicode
και ένα συνοδευτικό σύστημα ανάλυσης κώδικα Unicode Code (CPURS,
προφανείς "κατόχους"). Το CPURS θα αντικαταστήσει το DNS και θα παράσχει ένα
αρχιτεκτονική και μηχανισμό ανάλυσης για την επίλυση σημείων κώδικα Unicode
με τους εγγεγραμμένους χαρακτήρες και τις ετικέτες μικρών ονομάτων και αντίστροφα. ο
τα νέα μητρώα και καταχωρητές Unicode θα αντικαταστήσουν την κληρονομιά
τα αντίστοιχα ονόματα τομέα. Αυτό θα οδηγήσει σε μια νέα χρυσή βιασύνη για
καταχώρησης σημείων κώδικα Unicode για εταιρικά λογότυπα και προϊόντα
εικονογραφήσεις, και έτσι να αρχίσει μια εποχή οικονομικής ευημερίας, η οποία θα ήταν
τελικά να μειώσει την υπερθέρμανση του πλανήτη.




\subsection{Kaplan πληροφοριακό}


Μόλις δημιουργηθούν μητρώα Unicode και CPURS, θα χρησιμοποιηθούν οι διευθύνσεις IPv6
να κατανεμηθούν με την καταγραφή κωδικών μέσω αυτού του συστήματος · αυτοί
δεν θα καταχωρούνται πλέον από το IANA και το RIR. Αυτό δεν είναι σημαντικό
ανησυχία, ωστόσο, επειδή οποιαδήποτε απώλεια φορολογικών εσόδων θα είναι μεγαλύτερη από
αντισταθμίζεται από τα μητρώα Unicode που εκχωρούν σημεία κώδικα. Επιπλέον, στο
προκειμένου να καταστεί δυνατή η CPURS, τα πραγματικά γραφικά αρχεία για τα glyphs
πρέπει να τυποποιηθούν και να δημιουργηθούν σε πολλές μορφές και μεγέθη,
με διάφορους κανόνες πνευματικής ιδιοκτησίας. Αυτό θα προσφέρει περισσότερα
εργασία για γραφίστες και δικηγόρους, αυξάνοντας περαιτέρω τα φορολογικά έσοδα.

Ο ευφυής αναγνώστης μπορεί να ρωτήσει γιατί χρειαζόμαστε CPURS αν η μετάφραση Unicode
εκτελείται επί του παρόντος τοπικά στους οικοδεσπότες. Η απάντηση είναι τόμος: είναι
είναι απίθανο ότι οι εφαρμογές υποδοχής μπορούν να συμβαδίσουν με το ποσοστό νέων
Κωδικοί μονάδων Unicode διατίθενται για τα emojis, imojis και umojis.
Ενώ οι ενημερώσεις εφαρμογών και λειτουργικού συστήματος εμφανίστηκαν στο
ένα συνεχώς αυξανόμενο ποσοστό και σύντομα θα φτάσει στον ίδιο ρυθμό με τον άνθρωπο
γεννήσεις, είναι αμφίβολο ότι θα φτάσει ποτέ στο ρυθμό του αισθανόμενου
εξωγήινοι γεννήσεις. Επομένως, χρειαζόμαστε ένα σύστημα που μπορεί να κλιμακωθεί
να φτάσουμε σε αυτήν την ένταση πριν την πρώτη επαφή. διαφορετικά, το
διπλωματική αποτυχία να παράσχει γρήγορα στους εξωγήινους amojis του
μπορεί να οδηγήσει σε ένοπλες συγκρούσεις. Μια ένοπλη σύγκρουση με άλλους
αισθανόμενα όντα ικανά να φτάσουν στη Γη μπορεί να αυξηθούν σε παγκόσμιο επίπεδο
θέρμανσης, αποτρέποντας τον τελικό σκοπό αυτού του εγγράφου.

\section{Συμπέρασμα}
Η απλότητα των προηγούμενων πληροφοριών είναι ξεκάθαρη. Αυτό είναι το πιο απλό μάθημα.

\newpage
\section*{Πηγές}

	Copyright (c) 2014 Yusuke Endoh (@mametter), @hirekoke
	
	MIT License
	
	Permission is hereby granted, free of charge, to any person obtaining a copy of this software and associated documentation files (the "Software"), to deal in the Software without restriction, including without limitation the rights to use, copy, modify, merge, publish, distribute, sublicense, and/or sell copies of the Software, and to permit persons to whom the Software is furnished to do so, subject to the following conditions:
	
	The above copyright notice and this permission notice shall be included in all copies or substantial portions of the Software.
	
	THE SOFTWARE IS PROVIDED "AS IS", WITHOUT WARRANTY OF ANY KIND, EXPRESS OR IMPLIED, INCLUDING BUT NOT LIMITED TO THE WARRANTIES OF MERCHANTABILITY, FITNESS FOR A PARTICULAR PURPOSE AND NONINFRINGEMENT. IN NO EVENT SHALL THE AUTHORS OR COPYRIGHT HOLDERS BE LIABLE FOR ANY CLAIM, DAMAGES OR OTHER LIABILITY, WHETHER IN AN ACTION OF CONTRACT, TORT OR OTHERWISE, ARISING FROM, OUT OF OR IN CONNECTION WITH THE SOFTWARE OR THE USE OR OTHER DEALINGS IN THE SOFTWARE.


\url{https://tools.ietf.org/html/rfc8369}
\end{document}
