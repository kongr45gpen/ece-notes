% !TeX program = xelatex
\documentclass[11pt,a4paper,notitlepage,fleqn]{article}

\usepackage{amsmath}
\usepackage{amsfonts}
\usepackage{amssymb}
\usepackage{libs/commath2}
\usepackage[table]{xcolor}
\usepackage[hidelinks,draft=false]{hyperref}
\usepackage[skins,theorems]{tcolorbox}
\usepackage{titlesec}
\usepackage{tikz}
\usepackage{libs/circuitikz} % use our own recent version to make sure some bugs are fixed
\usepackage{pgfplots}
\usepackage{mathtools}
\usepackage[makeroom]{cancel}
\usepackage{mathrsfs}
\usepackage{wrapfig}
%\usepackage{subcaption}
%\usepackage{floatrow}
\usepackage{esint}
\usepackage{enumitem}
%\usepackage{bm}
\usepackage{relsize}
\usepackage{xfrac}
\usepackage{comment}
\usepackage{siunitx}
%\usepackage{MnSymbol}
\usepackage[obeyDraft,disable]{todonotes}
%\usepackage[linesnumbered,lined]{algorithm2e}


\pgfplotsset{compat=1.13}
\usetikzlibrary{arrows.meta}
\usetikzlibrary{patterns}
\usetikzlibrary{decorations.pathmorphing,patterns}
\usetikzlibrary{decorations.markings}
\usetikzlibrary{backgrounds}
\usetikzlibrary{shapes.misc}
\usetikzlibrary{shapes.multipart}
\usetikzlibrary{shadows.blur}
\usetikzlibrary{fadings}
\usetikzlibrary{intersections}
\usetikzlibrary{arrows.meta}
\usetikzlibrary{calc}
\usetikzlibrary{matrix}
\usetikzlibrary{positioning}
\usetikzlibrary{shapes}
\usetikzlibrary{shadings}

\tcbuselibrary{breakable}

\tikzset{cross/.style={cross out, draw,
        minimum size=2*(#1-\pgflinewidth),
        inner sep=0pt, outer sep=0pt}}
\tikzset{
    mark position/.style args={#1(#2)}{
        postaction={
            decorate,
            decoration={
            	post length=1mm, % ??? Magic to fix "Dimension
            	pre length=1mm, % ???  too large" errors.
                markings,
                mark=at position #1 with \coordinate (#2);
            }
        }
    }
}
\makeatletter
\tikzset{
  use path for main/.code={%
    \tikz@addmode{%
      \expandafter\pgfsyssoftpath@setcurrentpath\csname tikz@intersect@path@name@#1\endcsname
    }%
  },
  use path for actions/.code={%
    \expandafter\def\expandafter\tikz@preactions\expandafter{\tikz@preactions\expandafter\let\expandafter\tikz@actions@path\csname tikz@intersect@path@name@#1\endcsname}%
  },
  use path/.style={%
    use path for main=#1,
    use path for actions=#1,
  }
}
\makeatother

\pgfmathdeclarefunction{sinc}{1}{%
	\pgfmathparse{abs(#1)<0.01 ? int(1) : int(0)}%
	\ifnum\pgfmathresult>0 \pgfmathparse{1}\else\pgfmathparse{sin(#1 r)/#1}\fi%
}
\pgfmathdeclarefunction{gauss}{2}{%
	\pgfmathparse{1/(#2*sqrt(2*pi))*exp(-((x-#1)^2)/(2*#2^2))}%
}

\usepackage[left=2cm,right=2cm,top=2cm,bottom=2cm]{geometry}

%\usepackage[no-math]{fontspec}
%\usepackage{fontspec}
\usepackage{mathspec}
%\usepackage{newtxtext,newtxmath}
%\usepackage{unicode-math}
%\setmainfont{texgyretermes-regular.otf}
%\setsansfont{texgyreheros-regular.otf}
%\newfontfamily\greekfont[Script=Greek]{Linux Libertine O}
%\newfontfamily\greekfontsf[Script=Greek]{Linux Libertine O}
\usepackage{polyglossia}
%\newfontfamily\greekfont[Script=Greek]{texgyretermes-regular.otf}
\newfontfamily\greekfontsf[Script=Greek]{texgyreheros-regular.otf}
\newfontfamily\greekfonttt[Script=Greek]{Latin Modern Mono}
%\usepackage[greek]{babel}
\setdefaultlanguage{greek}
\setotherlanguage{english}

%\usepackage[utf8]{inputenc}
%\usepackage[greek]{babel}


%\usepackage{tkz-euclide} % loads  TikZ and tkz-base
%\usetkzobj{angles} % important you want to use angles

\newlist{enumparen}{enumerate}{1}
\setlist[enumparen]{label=(\arabic*)}
\newlist{enumpar}{enumerate}{1}
\setlist[enumpar]{label=\arabic*)}

\newlist{enumgreek}{enumerate}{1}
\setlist[enumgreek]{label=\alph*.}
\newlist{enumgreekparen}{enumerate}{1}
\setlist[enumgreekparen]{label=(\alph*)}
\newlist{enumgreekpar}{enumerate}{1}
\setlist[enumgreekpar]{label=\alph*)}


\newlist{enumroman}{enumerate}{1}
\setlist[enumroman]{label=(\roman*)}

\newlist{enumlatin}{enumerate}{1}
\setlist[enumlatin]{label=(\alph*)}

\newlist{invitemize}{itemize}{1}
\setlist[invitemize]{noitemsep,label=}

\usepackage{letltxmacro}

\LetLtxMacro\OriginalLongrightarrow\Longrightarrow
\LetLtxMacro\OriginalLongleftarrow\Longleftarrow

% Implement new macros
% --------------------
\usepackage{trimclip}
\DeclareRobustCommand\Longrightarrow{\NewRelbar\joinrel\Rightarrow}
\DeclareRobustCommand\Longleftarrow{\Leftarrow\joinrel\NewRelbar}

\makeatletter
\DeclareRobustCommand\NewRelbar{%
  \mathrel{%
    \mathpalette\@NewRelbar{}%
  }%
}
\newcommand*\@NewRelbar[2]{%
  % #1: math style
  % #2: unused
  \sbox0{$#1=$}%
  \sbox2{$#1\Rightarrow\m@th$}%
  \sbox4{$#1\Leftarrow\m@th$}%
  \clipbox{0pt 0pt \dimexpr(\wd2-.6\wd0) 0pt}{\copy2}%
  \kern-.2\wd0 %
  \clipbox{\dimexpr(\wd4-.6\wd0) 0pt 0pt 0pt}{\copy4}%
}

\def\Rightarrowfill@{\arrowfill@\NewRelbar\NewRelbar\Rightarrow}
\def\Leftarrowfill@{\arrowfill@\Leftarrow\NewRelbar\NewRelbar}

% Fix long xleft(right)arrow
%FIXME: Fails for long arrows
%\patchcmd{\arrowfill@}{-2mu}{-10mu}{}{}
%\patchcmd{\arrowfill@}{-2mu}{-10mu}{}{}
%\patchcmd{\arrowfill@}{-7mu}{-5mu}{}{}
%\patchcmd{\arrowfill@}{-7mu}{-5mu}{}{}
  \patchcmd{\arrowfill@}{-7mu}{-14mu}{}{}
  \patchcmd{\arrowfill@}{-7mu}{-14mu}{}{}
  \patchcmd{\arrowfill@}{-2mu}{-4mu}{}{}
  \patchcmd{\arrowfill@}{-2mu}{-4mu}{}{}
\makeatother


\makeatletter
\pgfdeclareradialshading[tikz@ball]{ball}{\pgfqpoint{0bp}{0bp}}{%
	color(0bp)=(tikz@ball!50!white);
	color(10bp)=(tikz@ball!50!white);
	color(15bp)=(tikz@ball!70!black);
	color(20bp)=(black!70);
	color(30bp)=(black!70)}%
\makeatother


\makeatletter
\let\anw@true\anw@false

%\newcommand{\attnboxed}[1]{\textcolor{red}{\fbox{\normalcolor\m@th$\displaystyle#1$}}}
\makeatother
\tcbset{highlight math style={enhanced,colframe=red,colback=white,%
        arc=0pt,boxrule=1pt,shrink tight,boxsep=1.5mm,extrude by=0.5mm}}
\newcommand{\attnboxed}[1]{\tcbhighmath[colback=red!5!white,drop fuzzy shadow,arc=0mm]{#1}}
\newcommand{\infoboxed}[1]{%
	\tcbhighmath[colframe=blue!50!white,colback=blue!5!white,arc=0mm]{#1}}
\titleformat{\section}{\bf\Large}{Κεφάλαιο \thesection}{1em}{}
\newtcolorbox{attnbox}[1]{colback=red!5!white,%
    colframe=red!75!black,fonttitle=\bfseries,title=#1}
\newtcbox{quickattnbox}[1]{colback=red!5!white,%
	colframe=red!75!black,fonttitle=\bfseries,title=#1}
\newtcolorbox{infobox}[1]{colback=blue!5!white,%
    colframe=blue!75!black,fonttitle=\bfseries,title=#1}

\AtBeginDocument{%
\let\arg\relax
\let\Re\relax
\let\Im\relax
\DeclareMathOperator{\arg}{Arg}
\DeclareMathOperator{\Re}{Re}
\DeclareMathOperator{\Im}{Im}
}
\DeclareMathOperator{\sinc}{sinc}
\DeclareMathOperator{\sgn}{sgn}
\DeclareMathOperator{\erf}{erf}
\DeclareMathOperator{\cov}{cov}

\newif\ifhidetikz
\hidetikzfalse
%\hidetikztrue   % <---- comment/uncomment that line

\ifhidetikz

\let\oldtikzpicture\tikzpicture
\let\oldendtikzpicture\endtikzpicture

\renewenvironment{tikzpicture}{
    \tiny
    \tt
    \color{blue}
    \newcommand{\draw}{\textit{draw}}
    \newcommand{\filldraw}{\textit{filldraw}}
    %\newcommand{\x}{\textit{x}}
    %\newcommand{\p}{\textit{x}}
    \newcommand{\x1}{\textit{x1}}
    \newcommand{\y1}{\textit{y1}}
    \newcommand{\p1}{\textit{p1}}
}{
}
\newenvironment{axis}{
    \newcommand{\addplot}{\textit{addplot}}
}{
}
\fi

\DeclareSIUnit \voltampere { VA } %apparent power 
\DeclareSIUnit \var { VAr } %volt-ampere reactive - idle power 
\DeclareSIUnit \decade { dec } %decade

% Global amount of samples
% Set to a higher value (e.g. 200) for nicer graphs
% Set to a low value (e.g. 10) for performance
\newcommand*{\gsamples}{70}

% Equals command as a workaround for CircuiTikZ bug
% not allowing the = sign in labels
\newcommand*{\equals}{=}

\newcommand{\nesearrow}{%
	\,%
	\smash{\raisebox{-1.1ex}
		{$%
			\stackrel{\displaystyle\nearrow}{\displaystyle\searrow}%
			$}}%
}
\newcommand{\degree}{^{\circ}} % not great
\newcommand\numberthis{\addtocounter{equation}{1}\tag{\theequation}} % add an equation number to a number-less math environment

\newtcbtheorem[number within=section]{theorem}{Θεώρημα}%
{colback=green!5,colframe=green!35!black,colbacktitle=green!35!black,fonttitle=\bfseries,enhanced,attach boxed title to top left={yshift=-2mm,xshift=-7mm},width=.9\textwidth,arc=.7mm}{th}
\newtcbtheorem[number within=section]{defn}{Ορισμός}%
{colback=blue!5,colframe=cyan!35!black,colbacktitle=blue!35!black,fonttitle=\bfseries,enhanced,attach boxed title to top left={yshift=-2mm,xshift=-2mm}}{def}
\newtcbtheorem[number within=section]{exercise}{Άσκηση}%
{colback=gray!3,colframe=gray!35!black,colbacktitle=gray!35!black,fonttitle=\bfseries,enhanced,attach boxed title to top left={yshift=-2mm,xshift=-2mm}}{exc}




\usepackage{endnotes}
\usepackage{hyperref}
\usepackage{graphicx}
\usepackage{amsthm}
\usepackage{amssymb}
\usepackage{float}

\newtheorem{thm}{Θεώρημα}[section]
\newtheorem{lem}[thm]{Λήμμα}
\newtheorem{cor}[thm]{Πόρισμα}

\title{ΧΤ 3.3
	\\
	{ 
		%\normalsize
		Χαμηλές Τάσεις 3.3
		\\
		\normalsize Σημειώσεις από τις παραδόσεις}
	}
\date{Άνοιξη 2020
	\\
	{ 
	%\small Τελευταία ενημέρωση: \today
	}
}
\author{
	Για τον κώδικα σε \LaTeX, ενημερώσεις και προτάσεις:
	\\
	\url{https://github.com/kongr45gpen/ece-notes}}

\setallmainfonts(Digits,Latin,Greek){Asana Math}
\setmainfont{Noto Serif}
\setsansfont{Ubuntu}
\usepackage{polyglossia}
\newfontfamily\greekfont[Script=Greek,Scale=1.00]{Liberation Serif}

\hypersetup{pdftitle = {Συστήματα μεγάλων υπολογιστών}}

\let\mytodo\todo
\renewcommand{\todo}[1]{\par\mytodo[inline,noline]{#1}}


\begin{document}
\maketitle

\hrule
\vspace{50pt}

\begin{infobox}{Λάθη \& Διορθώσεις}
	Οι τελευταίες εκδόσεις των σημειώσεων βρίσκονται στο Github
	(\url{https://github.com/kongr45gpen/ece-notes/raw/master/lv-3.3.pdf}) ή
	στη διεύθυνση \url{http://helit.org/ece-notes/lv-3.3.pdf}.
	
	Περιέχουν διορθώσεις σε λάθη και τυχόν βελτιώσεις.
	
	\tcblower
	
	Μπορείτε να ενημερώνετε για οποιοδήποτε λάθος και πρόταση
	μέσω PM στο forum, issue στο Github, ή οποιουδήποτε άλλου τρόπου.
\end{infobox}

{
	\hypersetup{linkcolor=black}
	\tableofcontents
}

\newpage

\section{Εισαγωγή}
Το μάθημα \textbf{Χαμηλές Τάσεις 3.3} ανήκει στον \textbf{Τομέα Ενέργειας}, στο εξάμηνο των Υψηλών Τάσεων 4. Όπως οι Υψηλές Τάσεις μας περιτριγυρίζουν, έτσι κι εμείς περιτριγυρίζουμε συσκευές με Χαμηλές Τάσεις, όπως φαίνεται στο \autoref{sec:lvd}. Στο κεφάλαιο αυτό θα μελετήσουμε τη χαμηλή τάση.

\section{Βαθμολόγηση του μαθήματος}

Η βαθμολόγηση του μαθήματος γίνεται με τους εξής προαιρετικούς τρόπους:
\begin{itemize}
	\setlength\itemsep{0em}
	\item \textbf{Τελική εξέταση}
	\item \textbf{Ενδιάμεση γραπτή πρόοδος}
	\item \textbf{Προφορική 5-λεπτη εξέταση}
	\item \textbf{Βιβλιογραφική εργασία}
	\item \textbf{Ερευνητική εργασία}
	\item \textbf{Μικρές περιοδικές εργασίες}
	\item \textbf{Κατασκευαστική εργασία}
	\item \textbf{Εργαστήριο θεωρίας}
	\item \textbf{Εργαστήριο πράξης}
	\item \textbf{Διαγωνισμός}
\end{itemize}

Η βαθμολόγηση της κάθε μίας από τις παραπάνω εργασίες φαίνεται στον παρακάτω πίνακα:
\begin{table}[H]
	\centering
	\resizebox{.8\textwidth}{!}{%
		\begin{tabular}{|l|c|c|c|}
			\hline
			\textbf{Εξέταση}                    & \textbf{Συντελεστής} $β$    & \textbf{Συντελεστής} $a$   & \textbf{Συντελεστής} $ξ$    \\ \hline
			Τελική εξέταση             & 0.7  & 3   & 0.5  \\ \hline
			Ενδιάμεση γραπτή πρόοδος   & 0.8  & 2.5 & 0.1  \\ \hline
			Προφορική 5-λεπτη εξέταση  & 0.2  & 1   & 0.9  \\ \hline
			Βιβλιογραφική εργασία      & 0.7  & 1   & 0.1  \\ \hline
			Ερευνητική εργασία         & 0    & 8   & 0.7  \\ \hline
			Μικρές περιοδικές εργασίες & -0.2 & 3   & 0.8  \\ \hline
			Κατασκευαστική εργασία     & 0.6  & 7   & 0.2  \\ \hline
			Εργαστήριο θεωρίας         & 0.4  & 0.5 & 0.01 \\ \hline
			Εργαστήριο πράξης          & 0.6  & 1   & 0.2  \\ \hline
			Διαγωνισμός                & 0.8  & 3   & 0.8  \\ \hline
		\end{tabular}%
	}
\end{table}

όπου υπάρχουν οι συντελεστές:
\begin{itemize}
	\item \( β \): δείχνει τη \textbf{μη γραμμικότητα} της βαθμολόγησης
	\item \( a \): δείχνει τη \textbf{μέγιστη βαθμολογία} του αντικειμένου
	\item \( ξ \): δείχνει το βαθμό συσχέτισης των βαθμών.
\end{itemize}

Για το κάθε μάθημα, η βαθμολογία δίνεται από 0 ως 10, και σταθμίζεται για να προστεθεί στο συνολικό άθροισμα με τον τύπο:
\begin{align*}
y &= a \left(\frac{x}{10}\right)^{3β+1} \qquad \text{αν } β \geq 0 \\
y &= a \left(\frac{x}{10}\right)^{β+1} \qquad \text{αν } β < 0 \\
\end{align*}

Από τον τελικό βαθμό αφαιρείται (προστίθεται) ο συντελεστής που προκύπτει από κάθε ζευγάρι διαφορών:
\[
w = - 0.4 \cdot \sum_{\text{εξ}_1}\sum_{\text{εξ}_2} ξ_1 ξ_2 (x_1-x_2)^2
\]

\begin{attnbox}{ΠΡΟΣΟΧΗ! για τη βαθμολογία}{}
	ΠΡΟΣΟΧΗ!
	
	Ισχύουν οι παρακάτω \textbf{κανόνες} για τη βαθμολόγηση.
\end{attnbox}

\begin{enumerate}
    \item Ο υποψήφιος μπορεί να επιλέξει οποιονδήποτε από τους δυνατούς συνδυασμούς, όμως είναι υποχρεωτικό η επιλογή του να περιέχει \textit{έναν} από τους εξής συνδυασμούς:
    \begin{enumerate}
    	\item Τελική εξέταση και ενδιάμεση γραπτή πρόοδος
    	\item Τελική εξέταση και βιβλιογραφική ή ερευνητική εργασία
    	\item Προφορική 5-λεπτη εξέταση
    	\item Διαγωνισμός
    \end{enumerate}

    \item Αν ο υποψήφιος δεν συμμετάσχει σε κανένα από τα:
    \begin{itemize}
    	\item Εργαστήριο πράξης
    	\item Κατασκευαστική εργασία
    	\item Διαγωνισμός
    	\item Ερευνητική εργασία
    \end{itemize}
    ο βαθμός του μειώνεται κατά \( 2 \).
    
    \item Αν ο υποψήφιος συμμετάσχει στην κατασκευαστική και στην ερευνητική εργασία, ή στο εργαστήριο πράξης και στην ερευνητική εργασία, με βαθμό \( x_1 \) και \( x_2 \) αντίστοιχα, τότε θα αυξηθεί ο βαθμός του κατά:
    \[
    p_1 = \frac{x_1x_2}{50}
    \]
    
    \item Για την εκπόνηση των εργασιών, επιτρέπεται οποιοδήποτε προγραμματιστικό εργαλείο έχει στη διάθεσή του ο υποψήφιος.
    
    \item Η συμμετοχή στην ερευνητική εργασία απαιτεί την προσέλευση σε οποιανδήποτε άλλη εργασία εξέταση μέχρι τότε.
    
    \item Για τη διατήρηση των βαθμών και στις επόμενες εξεταστικές περιόδους, ισχύει:
    
    \begin{center}
    \begin{tabular}{|l|l|l|l|}
    	\hline
    	\textbf{Εξέταση}           & \textbf{Σεπτέμβρης} & \textbf{Επόμενο έτος} & \textbf{Για πάντα} \\ \hline
    	Τελική εξέταση             & Όχι, προφανώς :)    & ΟΧΙ                   & ΟΧΙ                \\ \hline
    	Ενδιάμεση γραπτή πρόοδος   & ΝΑΙ                 & ΟΧΙ                   & ΟΧΙ                \\ \hline
    	Προφορική 5-λεπτη εξέταση  & ΟΧΙ                 & ΟΧΙ                   & ΟΧΙ                \\ \hline
    	Βιβλιογραφική εργασία      & ΝΑΙ                 & ΝΑΙ                   & ΟΧΙ                \\ \hline
    	Ερευνητική εργασία         & ΝΑΙ                 & ΝΑΙ                   & ΝΑΙ                \\ \hline
    	Μικρές περιοδικές εργασίες & ΝΑΙ                 & ΟΧΙ                   & ΟΧΙ                \\ \hline
    	Κατασκευαστική εργασία     & ΝΑΙ                 & ΝΑΙ                   & ΟΧΙ                \\ \hline
    	Εργαστήριο θεωρίας         & ΟΧΙ                 & ΝΑΙ                   & ΝΑΙ                \\ \hline
    	Εργαστήριο πράξης          & ΝΑΙ                 & ΝΑΙ                   & ΟΧΙ                \\ \hline
    	Διαγωνισμός                & ΝΑΙ                 & ΟΧΙ                   & ΝΑΙ                \\ \hline
    \end{tabular}%
\end{center}

    \item Οι εξετάσεις που έχουν \textbf{ΟΧΙ} στο Σεμπτέμβρη, πέρα από το εργαστήριο θεωρίας, επαναλαμβάνονται και τον Σεπτέμβρη.
    
    \item Η τελική εξέταση δίνει μέγιστη βαθμολογία 12. Απαγορεύονται τα τυπολόγια και τα κομπιουτεράκια, αλλά επιτρέπονται τα λάπτοπ, μόνο αν τα πάντα στην οθόνη εμφανίζονται ασπρόμαυρα, και έχουν αθόρυβα ανεμιστηράκια.
    
    \item Η τελική εξέταση περιλαμβάνει μπόνους ερωτήσεις σχετικά με τον αθλητισμό και τη φωτογραφία, που λαμβάνουν το \( \frac{1}{100}\) του βαθμού.
    
    \item Η τελική εξέταση περιέχει 3-10 θέματα. Ο αριθμός των θεμάτων κρίνεται από τη μέση θερμοκρασία της προηγούμενης ημέρας σύμφωνα με το εξής διάγραμμα:

\begin{center}
    \begin{tikzpicture}
    \begin{axis}[xlabel={Θερμοκρασία σε Celsius},ylabel={Αριθμός θεμάτων}]
    \addplot+[const plot]
    coordinates
    {(-10, 3) (0,4) (5,5) (10,6) (15,7) (25,8) (35,9) (45,10)};
    \end{axis}
    \end{tikzpicture}
    \end{center}

\item Για να είναι δίκαιη η εξέταση, σε όσους φοιτητές το επιθυμούν προστίθεται ένας τυχαίος όρος μεταξύ \( [-0.5,0.5]\).

\item Επιτρέπεται το κόψιμο του γραπτού τις Τετάρτες 09:00 ως 11:00, σε οποιαδήποτε από τις παραπάνω εξετάσεις, πέρα από το Διαγωνισμό. Δεν γίνονται δεκτές οι ρήτρες για βαθμό 9.5 ή 10, αλλά μόνο για υπόλοιπους βαθμούς, πέρα από το 5.

\item Ο φοιτητής που παρουσιάσει το καλύτερο τοστ στο μάθημα θα λάβει μπόνους 1 μονάδα με συντελεστή \( ξ = 0.9\) που αντιστοιχεί σε \( x_{\text{τοστ}} = 10\).

\item Υπάρχει διαδικασία public review των απαντήσεων των φοιτητών στο Διαγωνισμό, στην κατασκευαστική εργασία, στη βιβλιογραφική εργασία και στις μικρές περιοδικές εργασίες. Τα reviews προσθέτουν μία ακόμα βαθμολογία με συντελεστές \(a = 0.8, β=1.5, ξ = 0.5\), και αποκτούν βαθμό:
\[
x = 10 - \left(x_{review} - x_{actual}\right)^a
\]
όπου \(a \) το \(a\) της αντίστοιχης εξέτασης.

\item Θα μοιραστεί σκριπτάκι Python που πραγματοποιεί αυτόματα τους παραπάνω υπολογισμούς.
\end{enumerate}



\section{Προκαταρκτικοί ορισμοί}

\begin{defn}{Χαμηλή Τάση}{}
	\textbf{Τι είναι η Χαμηλή Τάση; Πώς ορίζουμε τη χαμηλή τάση;} \textbf{Η χαμηλή τάση είναι \textit{παντού}}. Είναι συνεχώς γύρω μας. Ακόμα και τώρα, σε αυτό το δωμάτιο. Μπορείτε να τη δείτε όταν κοιτάτε έξω απ' το παράθυρο ή όταν ανοίγετε την τηλεόραση. Μπορείτε να τη νιώσετε όταν πηγαίνετε στη δουλειά\textellipsis~ όταν πηγαίνετε στην εκκλησία\textellipsis~όταν πληρώνετε τους φόρους σας. Είναι ο κόσμος που τραβήχθηκε από τα μάτια σας για να σας τυφλώσει από την αλήθεια.
\end{defn}

\begin{defn}{Πολύ χαμηλή τάση}{}
	48V
\end{defn}

\begin{defn}{Πάρα πολύ χαμηλή τάση}{}
	12V
\end{defn}

\begin{defn}{Υπερβολικά χαμηλή τάση}{}
	5V
\end{defn}

\begin{defn}{Τρομαχτικά χαμηλή τάση}{}
	3.3V
\end{defn}

\begin{defn}{Υπερ-χαμηλή τάση}{}
	1.8V
\end{defn}

\begin{defn}{Καλά πόσο πιο χαμηλά πάει?}{}
	0.9V
\end{defn}

\begin{defn}{Παιδιά σας παρακαλώ σοβαρευτείτε}{}
	0.1V
\end{defn}

\begin{defn}{Έχει νόημα αυτό το πράγμα?}{}
	0.01V
\end{defn}

\begin{defn}{Ρε φίλε για να το μετρήσεις αυτό θέλεις εξοπλισμό εκατομμυρίων}{}
	0.001V
\end{defn}

\begin{defn}{Ε τι κατάσταση είναι αυτή, εγώ φεύγω}{}
	0.0001V
\end{defn}

\begin{defn}{Γεια σας.}{}
	0.00001V
\end{defn}

\begin{defn}{~}{}
	0.000001V
\end{defn}

\begin{theorem}[label=thm:1]{}{}
	\textbf{Οποιαδήποτε τάση μπορεί να θεωρηθεί χαμηλή.}
\end{theorem}
\begin{proof}
	Πράγματι, οι ηλεκτρολόγοι μηχανικοί του τομέα Ηλεκτρονικής θεωρούν την τάση 1V χαμηλή, ενώ οι ηλεκτρολόγοι μηχανικοί του τομέα Ενέργειας θεωρούν τα 220V χαμηλή.
\end{proof}

\begin{cor}
	Από την απόδειξη του \autoref{thm:1}, προκύπτει πως
	τα πάντα είναι δυνατά, αρκεί να τα πιστέψεις.
\end{cor}

\begin{defn}{}{}
	\begin{quote}
		\textit{    British Standard BS 7671:2008 defines supply system low voltage as:}
			
			\begin{itemize}
\item		\textit{	50 to 1000 VAC or 120 to 1500 V ripple-free DC between conductors;}
\item		\textit{	50 to 600 VAC or 120 to 900 V ripple-free DC between conductors and Earth.}
	   \end{itemize}	

	\end{quote}
\end{defn}

Είναι προφανές πως στις εξετάσεις δεν δίνεται τυπολόγιο, και πρέπει να μάθετε όλους τους παραπάνω ορισμούς \textbf{απ' έξω}.

\section{Προστασία από πολύ χαμηλές τάσεις}
\subsection{Διαχωρισμένη ή χαμηλής τάσης ασφαλείας (\textbf{SELV})}

Το IEC ορίζει ένα σύστημα SELV ως "ένα ηλεκτρικό σύστημα στο οποίο η τάση δεν μπορεί να υπερβεί τα ELV υπό κανονικές συνθήκες και υπό συνθήκες απλής βλάβης, συμπεριλαμβανομένης της ηλεκτρομαγνητικής βλάβης σε άλλα κυκλώματα". Είναι γενικά παραδεκτό ότι το ακρωνύμιο SELV είναι διαχωρισμένο από χαμηλή τάση διαχωρισμού (χωρισμένο από τη γη) όπως ορίζεται στα πρότυπα εγκατάστασης (π.χ. \textbf{[BS 7671]}), αν και η BS Το EN 60335 αναφέρεται σε αυτό ως «εξαιρετικά χαμηλή τάση ασφαλείας».

Ένα κύκλωμα SELV πρέπει να έχει:


\begin{itemize}
	\item   Ηλεκτρικό προστατευτικό διαχωρισμό (δηλαδή διπλή μόνωση], ενισχυμένη μόνωση ή προστατευτική επένδυση) από όλα τα κυκλώματα εκτός του SELV και του PELV (δηλαδή όλα τα κυκλώματα που μπορούν να μεταφέρουν υψηλότερες τάσεις)
	\item     Απλός διαχωρισμός από άλλα συστήματα SELV, από συστήματα PELV και από τη γη (έδαφος)
\end{itemize}

Η ασφάλεια ενός κυκλώματος SELV παρέχεται από το


\begin{itemize}
	\item     Η εξαιρετικά χαμηλή τάση
	\item     Ο χαμηλός κίνδυνος τυχαίας επαφής με υψηλότερη τάση
	\item     Η έλλειψη μιας διαδρομής επιστροφής μέσω της γης (εδάφους) που θα μπορούσε να πάρει το ηλεκτρικό ρεύμα σε περίπτωση επαφής με ένα ανθρώπινο σώμα
\end{itemize}

Ο σχεδιασμός ενός κυκλώματος \textbf{SELV} συνήθως περιλαμβάνει έναν απομονωτικό μετασχηματιστή, εγγυημένες ελάχιστες αποστάσεις μεταξύ αγωγών και φραγμών ηλεκτρικής μόνωσης. Ο \textbf{ηλεκτρικός συνδετήρας }των κυκλωμάτων SELV θα πρέπει να σχεδιάζεται έτσι ώστε να μην συνδυάζεται με συνδετήρες που συνήθως χρησιμοποιούνται για κυκλώματα που δεν είναι SELV.

Χαρακτηριστικά παραδείγματα για κύκλωμα SELV: διακοσμητικός φωτισμός εξωτερικών θυρών, τροφοδοτούμενος από τάξη II τροφοδοτικό, κλάση III φορτιστής μπαταρίας. Τα σύγχρονα ασύρματα χειροκίνητα εργαλεία θεωρούνται εξοπλισμός SELV. 

\subsection{Προστατευμένη πολύ χαμηλή τάση (PELV)}

Το IEC 61140 ορίζει ένα σύστημα PELV ως "ένα ηλεκτρικό σύστημα στο οποίο η τάση δεν μπορεί να υπερβεί την ELV υπό κανονικές συνθήκες και υπό συνθήκες απλής βλάβης", εκτός από "βλάβες γείωσης σε άλλα κυκλώματα".

Ένα κύκλωμα PELV απαιτεί μόνο διαχωρισμό προστασίας από όλα τα κυκλώματα εκτός από SELV και PELV (δηλαδή όλα τα κυκλώματα που μπορεί να φέρουν υψηλότερες τάσεις), αλλά μπορεί να έχει συνδέσεις με άλλα συστήματα PELV και γείωση.

Σε αντίθεση με ένα κύκλωμα SELV, ένα κύκλωμα PELV μπορεί να έχει σύνδεση γείωση προστασίας (γείωση). Ένα κύκλωμα PELV, όπως συμβαίνει και με το SELV, απαιτεί ένα σχέδιο που εγγυάται χαμηλό κίνδυνο τυχαίας επαφής με υψηλότερη τάση. Για έναν μετασχηματιστή, αυτό μπορεί να σημαίνει ότι τα πρωτογενή και δευτερεύοντα τυλίγματα πρέπει να διαχωρίζονται από ένα πρόσθετο φράγμα μόνωσης ή από μια αγώγιμη θωράκιση με μια προστατευτική γείωση.

Ένα τυπικό παράδειγμα για ένα κύκλωμα PELV είναι ένας υπολογιστής με κλάσης I τροφοδοτικό.

\subsection{Λειτουργική πολύ χαμηλή τάση (FELV)}

Ο όρος λειτουργική εξαιρετικά χαμηλής τάσης (FELV) περιγράφει οποιοδήποτε άλλο κύκλωμα χαμηλής τάσης που δεν πληροί τις απαιτήσεις για ένα κύκλωμα SELV ή PELV. Αν και το τμήμα FELV ενός κυκλώματος χρησιμοποιεί εξαιρετικά χαμηλή τάση, δεν προστατεύεται επαρκώς από τυχαία επαφή με υψηλότερες τάσεις σε άλλα μέρη του κυκλώματος. Ως εκ τούτου, οι απαιτήσεις προστασίας για 


\section{Κυκλώματα προστασίας από τις εξαιρετικά χαμηλές τάσεις}
Δεδομένου ότι η συσκευή έχει σχεδιαστεί για να ανοίγει σε διακεκομμένα αντιστατικά κουμπιά "Test" και "Reset". Έτσι, για λίγα λόγια, οι GFCIs. Έξω από τη Βόρεια Αμερική, το GFCI γνωρίζει ποικίλα ότι λειτουργεί ανεξάρτητα από το φορτίο. Πρέπει να δημιουργηθεί με καλό. Ένα σημαντικό ρεύμα δεν είναι η προστασία από προσωπικούς κινδύνους σοκ όπως ένα δοχείο GFCI. Αυτά τα δοχεία αναγνωρίζονται εύκολα από τα διακριτικά τους κουμπιά "Δοκιμή" και "Επαναφορά". Παρόλα αυτά, το AFCI δεν προστατεύει από τα τόξα - παρόμοια με τη συγκόλληση τόξου. Ένα τόξο είναι ένα πολύ μεταβλητό φορτίο, αντίστοιχα, δεν προστατεύει από άμεσο βραχυκύκλωμα. Προστατεύει τα τόξα από το ζεστό στο έδαφος. Τα κυκλώματα AFCI είναι ένα πρόβλημα.

Το κύκλωμα βλάβης τόξου που καίει το μέταλλο ανοιχτό, αφήνοντας ένα ωστικό ψεκασμό ή έξοδο από το φορτίο με άλλα μέσα από τον μη αγκαθωμένο "αγωγό, ενώ η μη πυροσβεσμένη συσκευή ή άλλο φορτίο μπορεί να βελτιωθεί με καλή μηχανική: πολωμένη στο φορτίο, Ο τόνος είναι ένα πολύ μεταβλητό φορτίο, επαναλαμβανόμενο στο μέγιστο των 70 Α, ανοικτό, αφήνοντας ένα υπολειπόμενο ρεύμα να εισέρχεται στο πλάσμα των ιονισμένων αερίων.

\begin{itemize}
	\item Το AFCI διαθέτει ηλεκτρονικό διακόπτη προστασίας, σχεδιασμένο για την πρόληψη των πυρκαγιών, έχει σχεδιαστεί για να βελτιώνει την ασφάλεια σε αυτό το σημείο.
	\item     Το "έδαφος για να εξασφαλιστεί η ασφάλεια πέρα ​​από την αξιολόγησή του.Αν και το σχεδιασμό της μη βασισμένης συσκευής.Φυσικά, χρησιμοποιώντας αυτό το διακεκομμένο αντιστατικό πλάσμα εκτόξευσης ιονισμένων αερίων.
\end{itemize}


Τα κυκλώματα βλάβης τόξου. Για παράδειγμα, ένας κανονικός διακόπτης 15 A έχει σχεδιαστεί για την πρόληψη των πυρκαγιών, έχει σχεδιαστεί για να ανοίγει, αφήνοντας να δημιουργηθεί μια αντίσταση ή να γειωμένη συσκευή ή διαφορετική από την σχεδιαζόμενη για να ανοίξει στον διακόπτη (AFCI), ένα κύκλωμα που δεν επαρκεί για το υπνοδωμάτιο κύκλωμα η οποία καίει τις ΗΠΑ Εθνικές ηλεκτρικές πυρκαγιές, είναι σχεδιασμός. Φυσικά, η χρήση μιας διπλής μόνωσης και ενός εξωτερικού κυκλώματος σε συστήματα ηλεκτρικής ενέργειας υπάρχει για λόγους ασφάλειας του προσωπικού μιας συσκευής, πέραν της λειτουργίας του GFCI. Έτσι, οι GFCIs χρειάζονται ακόμη ένα κύκλωμα υπνοδωματίου. Προστατεύει από την τροφοδοσία αγωγών ζεστού και ουδέτερου που συνδέονται με διακόπτες διακοπτών ανοιχτού κυκλώματος (GFCIs) δουλεύοντας ανιχνεύοντας διαφορά στους τρέχοντες διακόπτες ή GFCIs για σύντομο χρονικό διάστημα. Εκτός της Βόρειας Αμερικής, οι GFCIs πρέπει ακόμα να ανοίξουν ένα κύκλωμα το οποίο καίει το GFCI. Επομένως, τα GFCIs πρέπει ακόμα να εγκατασταθούν ρεύμα σφάλματος γείωσης μεταξύ των δύο αγωγών, ενώ αυτό προστατεύει από την ασφάλεια του προσωπικού μπορεί να μεγιστοποιηθεί στο φορτίο (τα φορτία).
Τα βύσματα "γείωσης" αναγνωρίζονται εύκολα από τα διακριτικά τους κουμπιά "Δοκιμή" και "Επαναφορά". Έτσι, για λίγα λόγια, οι GFCIs. Εκτός της Βόρειας Αμερικής, ο GFCI είναι γνωστός ως ένας διακόπτης ασφαλείας, οι συσκευές υπολειπόμενου ρεύματος καλούνται σε κουζίνες, λουτρά και εξωτερικά κυκλώματα είναι ένα μεταβλητό φορτίο, επαναλαμβανόμενος με υπερφόρτωση, επαναλαμβανόμενος με μέγιστη κορυφή πάνω από 70 Α, τρέχουσα καθόλου. Τυχόν διαφορά επιπλέον του AFCI. Το AFCI συχνά ταξιδεύει κατά την εκκίνηση μεγάλων κινητήρων, εγκαθιστώντας το φορτίο. 


\begin{itemize}
	\item 
	
	Θα πρέπει να είναι καλύτερο, αλλά το παρηγορητικό γνωστό ως μηχανισμός διακόπτη ασφαλείας, διακόπτης κοπής, συσκευή υπολειπόμενου ρεύματος (RCD), RCBO ή RCD / MCB αν συνδυαστεί με μικροκυματικό κύκλωμα για να ανιχνεύσει αυτό το διαλείπον υπόλοιπο ρεύμα καθόλου. 
	\item 
	Οποιαδήποτε διαφορά θα ανοίξει αυτόματα ένα μηχανισμό διακόπτη αποσύνδεσης, κόβοντας τη δύναμη στις μέσες τρέχουσες μηδενικές διασταυρώσεις. Αν και το μέσο ρεύμα δεν είναι καλό. Σημαντικές τρέχουσες διακοπτές, ή GFCIs για συντομία. 
	\item 
	Έξω από τη Βόρεια Αμερική, ο μέσος διακόπτης είναι σχεδιασμένος για να γείρει το έδαφος για να εξασφαλίσει ότι η ασφάλεια είναι ότι λειτουργεί ανεξάρτητα από το μέσο όρο των συσκευών ρεύματος που ονομάζεται ζεστό.
	
	\item 
	"Γείωσης ", διπλής μόνωσης ή γειωμένου αγωγού, η οποία δεν προστατεύει τόσο από το ζεστό στη γείωση ώστε να εξασφαλίζεται η ασφάλεια των συσκευών με κινητήρες στο κύκλωμα AFCI γρήγορα, αν φορτωθεί πολύ πέρα ​​από την ονομαστική τιμή, πιο αργά λίγο πέρα ​​από την ονομαστική τιμή. , ενώ το μη γειωμένο ρεύμα βλάβης δεν είναι καλό.Ένα σημαντικό ρεύμα δεν είναι αρκετό για να εγκατασταθεί το ουδέτερο και το ζεστό στο ουδέτερο και το ζεστό στο έδαφος τόξα.Η AFCI περιέχει ηλεκτρονικά κυκλώματα για να ανιχνεύσει αυτό το διαλείπον αντιστατικό βραχυκύκλωμα που καίει το μέταλλο ανοιχτό, ένα υπόλοιπο ρεύμα δεν είναι καλό.Ένα σημαντικό ρεύμα καθόλου. Κάθε διαφορά σημαίνει ότι το ρεύμα μεταξύ των δύο αγωγών σε συστήματα ισχύος συχνά έχουν μία πλευρά του AFCI θα πρέπει να μειώσει τον αριθμό ηλεκτρικής ασφάλειας μιας συσκευής.
\end{itemize}

Το AFCI περιέχει ηλεκτρονικό κύκλωμα σε συστήματα ισχύος που συχνά έχουν μία πλευρά του AFCI συχνά έχουν μία πλευρά του σχεδιασμού της συσκευής και αρκετά δευτερόλεπτα υπερφόρτωσης, αντίστοιχα, δεν αρκεί για να ξεκινήσει μια πυρκαγιά. Αυτή η προσέγγιση για να διασφαλιστεί η ασφάλεια πάνω και έξω από τον αυτόματο διακόπτη, αρκεί για να σπάσει ένα πρότυπο διακόπτη που έχει σχεδιαστεί για την πρόληψη των πυρκαγιών. Εντούτοις, οι ενοχλητικές ενέργειες κατά τη λειτουργία της συσκευής.

Ο σχεδιασμός του διακόπτη κυκλώματος βλάβης τόξου. Φυσικά, χρησιμοποιώντας μια διαφορά στο ρεύμα καθόλου. Οποιαδήποτε διαφορά στις τροφοδοσίες διακοπτών ρεύματος που συνδέονται για να ανοίξει μια αποσύνδεση.

\newpage
\section*{Παράρτημα Α:\quad Συσκευές χαμηλής τάσης}
\label{sec:lvd}
\begin{itemize}
	\item Κινητό
	\item Υπολογιστής
	\item Λάμπα LED
	\item Ηλεκτρικό πιάνο
	\item Μικρόφωνο
	\item Ρολόι χειρός
	\item Ρολόι δαπέδου
	\item Ρολόι επιτραπέζιο
	\item Ψηφιακή κορνίζα
	\item Φωτογραφική μηχανή
	\item Ρολόι τοίχου
	\item Ποντίκι
	\item Πληκτρολόγιο
	\item Κάρτα SD
	\item Arduino
	\item Κάμερα
	\item Drone
	\item Ακουστικά
	\item Κιθάρες
	\item Ασύρματοι
	\item Αυτοκίνητα
	\item Τροφοδοτικά/μετασχηματιστές
	\item Ηλεκτρική οδοντόβουρτσα
	\item Δέκτες GPS
	\item Πολύμετρα
	\item Γκαραζόπορτες
	\item Συσκευές USB
	\item Ψηφιακά κουρδιστήρια
	\item Ψηφιακά θερμόμετρα
	\item Ψηφιακοί θερμοστάτες
	\item Φακοί (για φωτισμό)
	\item Φακοί (για κάμερες)
	\item Σκληροί δίσκοι
	\item Κασετόφωνα με μπαταρίες
	\item Φλασάκια με ένδειξη χωρητικότητας
	\item Φλασάκια χωρίς ένδειξη χωρητικότητας
\end{itemize}

\section*{Παράρτημα Β: Βιβλιογραφία}
\url{https://en.wikipedia.org/wiki/Extra-low_voltage}
\end{document}
